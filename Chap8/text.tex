\documentclass[a4paper]{jreport}
\usepackage{amsmath,amssymb}
\usepackage[dvipdfmx]{graphicx}
\usepackage[dvipdfm]{hyperref}
\usepackage{pxjahyper}
\usepackage{framed}
\usepackage{color}

\usepackage[top=20truemm,bottom=20truemm,left=15truemm,right=15truemm]{geometry}


\def\qedsymbol{$\square$}
\def\proofname{\gt{証明}\;}
\newenvironment{Proof}{\par\noindent{\it\proofname}}{{\unskip\nobreak\hfill{\it\qedsymbol}}\par\vskip 9pt}
\newenvironment{Proof*}{\par\noindent}{{\unskip\nobreak\hfill{\it\qedsymbol}}\par\vskip 9pt}
\ifx\undefined\bysame \newcommand{\bysame}{\leavevmode\hbox to3em{\hrulefill}\,}\fi
%
\numberwithin{equation}{section}
\newtheorem{Thm}     {定理}[section]
\newtheorem{Lemma}   [Thm]{補題}
\newtheorem{Def}     [Thm]{定義}
\newtheorem{Prop}    [Thm]{命題}
\newtheorem{Fact}    [Thm]{事実}
\newtheorem{Cor}     [Thm]{系}  
\newtheorem{Conj}    [Thm]{予想}
\newtheorem{Ex}      [Thm]{例}  
\newtheorem{Achiom}   [Thm]{公理}
\newtheorem{Method}[Thm]{方法} 
\newtheorem{Rem}  [Thm]{注意}
\newtheorem{Notation}[Thm]{記法}
\newtheorem{Symbol}  [Thm]{記号}
\newtheorem{Prob}    [Thm]{問題}
\makeatletter
\renewenvironment{leftbar}{%
  \def\FrameCommand{\vrule width 1pt \hspace{10pt}}% 
  \MakeFramed {\advance\hsize-\width \FrameRestore}}%
 {\endMakeFramed}
\makeatother

\newenvironment{redleftbar}{%
  \def\FrameCommand{\textcolor{red}{\vrule width 1pt} \hspace{10pt}}% 
  \MakeFramed {\advance\hsize-\width \FrameRestore}}%
 {\endMakeFramed}

\newenvironment{lightgrayleftbar}{%
  \def\FrameCommand{\textcolor{lightgray}{\vrule width 1zw} \hspace{10pt}}% 
  \MakeFramed {\advance\hsize-\width \FrameRestore}}%
{\endMakeFramed}
\def\C{\mathbb C}
\def\N{\mathbb N}
\def\Z{\mathbb Z}

\def\R{\mathbb R}
\def\F{\mathcal F}
\def\method{\begin{leftbar}\begin{Method}}
\def\methodx{\end{Method}\end{leftbar}}
\def\thm{\begin{leftbar}\begin{Thm}}
\def\thmx{\end{Thm}\end{leftbar}}
\def\prop{\begin{Prop}}
\def\propx{\end{Prop}}
\def\defb{\begin{Def}}
\def\defe{\end{Def}}
\def\defx{\end{Def}}
\newcommand{\supp}{\mathop{\mathrm{supp}}\nolimits}
\def\rem{\begin{Rem}}
\def\remx{\end{Rem}}
\def\prob{\begin{Prob}}
\def\probx{\end{Prob}}
\def\lem{\begin{Lemma}}
\def\lemx{\end{Lemma}}
\def\ex{\begin{Ex}}
\def\exx{\end{Ex}}
\def\cor{\begin{Cor}}
\def\corx{\end{Cor}}
\def\proof{\begin{Proof}}
\def\proofx{\end{Proof}}
\def\eq{\begin{equation}}
\def\eqx{\end{equation}}
\def\eqa{\begin{eqnarray}}
\def\eqax{\end{eqnarray}}
\def\a{\alpha}
\def\lmd{\lambda}
\def\omg{\omega}
\def\Lmd{\Lambda}
\def\Omg{\Omega}
\newcommand{\Image}{\mathop{\mathrm{Im}}\nolimits}
\newcommand{\Ker}{\mathop{\mathrm{Ker}}\nolimits}
\newcommand{\Coker}{\mathop{\mathrm{Coker}}\nolimits}
\newcommand{\vol}{\mathop{\mathrm{vol}}\nolimits}
\newcommand{\sgn}{\mathop{\mathrm{sgn}}\nolimits}
\title{Numerical Solution of Stochastic Differential Equations}
\usepackage{makeidx}
\makeindex

\begin{document}
\maketitle
\tableofcontents
\leavevmode \\
\chapter{決定論的微分方程式の時間離散近似}
\section{Introduction}
\eq
\label{eq1}
\dot{x} = \frac{dx}{dt} = a(t,x), \  \  x(t_0) = x_0
\eqx
というような(決定論的)微分方程式の初期値問題を解くことは多くの場面において必要となるが,一般にこの微分方程式の解$x = x(t;t_0,x_0)$を明示的に求めることは出来ない.また解が見つかったとしても,明示的でなかったり数値解析の視点からすると複雑で計算や描画に不向きであることが多い.\par
ここで広く使われている方法が{\bf 時間離散近似} である.
時間離散近似は,与えられた時間離散化 $ t_0 < t_1 < t_2 < \cdots < t_n < \cdots $に対して,
解$x(t_1;t_0,x_0),x(t_2;t_0,x_0),\cdots,x(t_n;t_0,x_0),\cdots$を近似するような,
$y_1,y_2,\cdots,y_n,\cdots$を生成する方程式である.この章ではまずこれらの近似方法の一覧を示し,次にその離散化誤差,整合性,安定性,収束といった性質を示す.
\section{時間離散近似方法}
(\ref{eq1})
のような初期値問題を解くときに一番シンプルな方法が以下のEuler法である.
\method[Euler法]
{\bf Euler法}は $ t_0 < t_1 < t_2 < \cdots < t_n < \cdots $ という間隔 $\Delta_n = t_{n+1} - t_{n}$ の離散化に対して,近似解を
\eq
\label{eq2}
y_{n+1} = y_n + a(t_n,y_n)\Delta_n,\  \  y_0 = x_0
\eqx
によって与える.
\methodx
次に,このEuler法を更に精度を上げるために以下の台形法を用いる.
\method[台形法]
{\bf 台形法}は $ t_0 < t_1 < t_2 < \cdots < t_n < \cdots $ という間隔 $\Delta_n = t_{n+1} - t_{n}$ の離散化に対して,近似解を
\eq
\label{eq3}
y_{n+1} = y_n + \frac{1}{2} \{ a(t_n,y_n) + a(t_{n+1},y_{n+1}) \} \Delta_n,\  \  y_0 = x_0
\eqx
によって与える.
\methodx
この方法は$y_{n+1}$という不定値が両辺に含まれているため{\bf implicit}な方法と呼ばれる.一般にはこの不定値は移行することが出来ないので計算が複雑になることがある.故にそれを修正したのが,以下の修正台形法である.
\method[修正台形法]
{\bf 修正台形法}は $ t_0 < t_1 < t_2 < \cdots < t_n < \cdots $ という間隔 $\Delta_n = t_{n+1} - t_{n}$ の離散化に対して,近似解を
\eqa
\label{eq4}
\bar{y}_{n+1} &=& y_n + a(t_n,y_n)\Delta_n  \\
\label{eq5}
y_{n+1} &=& y_n + \frac{1}{2} \{ a(t_n,y_n) + a(t_{n+1},\bar{y}_{n+1}) \} \Delta_n 
\eqax
によって与える.つまり,$\bar{y}_{n+1}$を下の式に代入して,
\eq
\label{eq6}
y_{n+1} = y_n + \frac{1}{2} \{ a(t_n,y_n) + a(t_{n+1},y_n + a(t_n,y_n)\Delta_n) \} \Delta_n
\eqx
によって与えられる.
\methodx
修正台形法は,{\bf improved Euler法}や{\bf Heun法}とも呼ばれている.
また,このような方法は{\bf 予測子修正子法}と呼ばれている.というのは,
一度(\ref{eq4})で予測子$\bar{y}_{n+1}$を計算し,それを(\ref{eq5})のような式に代入して修正子$y_{n+1}$を得ているからである.\par
ここで今まで上げたような方法を一般化して次の方法を得る.
\method[一段法]
{\bf 一段法}は $ t_0 < t_1 < t_2 < \cdots < t_n < \cdots $ という間隔 $\Delta_n = t_{n+1} - t_{n}$ の離散化に対して,近似解を
\eq
\label{eq7}
y_{n+1} = y_n + \Psi(t_n,y_n,\Delta_n) \Delta_n,\  \  y_0 = x_0
\eqx
によって与える.$\Psi(t,y,\Delta)$のことをincrement functionという.
\methodx
例えばEuler法(\ref{eq2})では$\Psi(t,y,\Delta) = a(t,x)$であり,Heun法(\ref{eq6})では
$\Psi(t,y,\Delta) = \frac{1}{2} \{ a(t,y) + a(t+\Delta ,y + a(t,y)\Delta) \}$であった.これまで上げた方法は全て一段法であったが,それとは別の多段法を紹介する.
\method[Adams-Bashford法]
{\bf Adams-Bashford法}は$ t_0 < t_1 < t_2 < \cdots < t_n < \cdots $ という間隔 $\Delta = t_{n+1} - t_{n}$ が一定な離散化に対して,近似解を
\eq
\label{eq8}
y_{n+1} = y_n + \frac{1}{12} \{ 23 a(t_n , y_n) - 16 a(t_{n-1},y_{n-1}) + 5 a(t_{n-2},y_{n-2}) \} \Delta
\eqx
によって与える.
\methodx
これは,$y_{n+1}$を求めるために$y_n , y_{n-1} , \cdots , y_{n-k} $を必要としているために多段法と呼ばれている.例えばAdams-Bashford法は3段法である.3段法において,計算を始めるためには,$y_0,y_1,y_2$を求める必要があるが,これらは一段法によって計算される.\par
また更に高い精度を得るために次のように,既存の方法を反復するような方法を紹介する.

\method[Richardson反復法]
間隔$\Delta = T/N$によって$[0,T]$が等間隔に$N$等分されており,その離散化に対してEuler法を適用する場合を考える.
$y_N(\Delta)$を$\Delta$間隔で離散化したときの近似解の$\Delta N$での値とする.
$x(T)$を$T$での真の解の値とする.このとき
\eq
y_N(\Delta) = x(T) + e(T) \Delta + O(\Delta^2)
\eqx
が成り立っており,また,$2N$等分して離散化したときのことを考えると,
\eq
y_{2N}(\frac{1}{2}\Delta) = x(T) + \frac{1}{2} e(T) \Delta + O(\Delta^2)
\eqx
が成り立っている.$e(T)$を消去することによって,
\eq
x(T) = 2y_{2N}(\frac{1}{2}\Delta) - y_N(\Delta) + O(\Delta^2)
\eqx
を得るので,
\eq
Z_N(\Delta) = 2y_{2N}(\frac{1}{2}\Delta) - y_N(\Delta)
\eqx
とすることによって近似を得る.これをRichardson反復法という.
\methodx
ここで,$a$に条件を課せば,次のTaylorの定理が成り立っていることに着目する.
\thm[Taylorの定理]
$x(t)$は$p+1$回連続微分可能であるとする.このとき,
$ t_0 < t_1 < t_2 < \cdots < t_n < \cdots $ という間隔 $\Delta_n = t_{n+1} - t_{n}$ の離散化に対して,
\eq
\label{eq21}
x(t_{n+1}) = x(t_n) + \frac{dx}{dt}(t_n)\Delta_n + \cdots + 
\frac{1}{p!}\frac{d^px}{dt^p}(t_n)\Delta^p_n + \frac{1}{(p+1)!}\frac{d^{p+1}x}{dt^{p+1}}(\theta_n)\Delta^{p+1}_n
\eqx
を満たすような$t_n < \theta_n < t_{n+1}$が存在する.
\thmx
そして,微分方程式
\eq
\label{eq22}
\dot{x} = \frac{dx}{dt} = a(t,x(t))
\eqx
に着目して,チェインルールを適応すれば,
\eq
\frac{dx}{dt} = a ,\frac{d^2x}{dt^2} = a_t + a_x a, 
\frac{d^3x}{dt^3} = a_{tt} + 2a_{tx}a + a_{xx}a^2 + a_ta_x +a_x^2 a, \cdots
\eqx
($t,x$での偏微分を省略して$a_t,a_x$と書いた)が成り立っているので,これらを代入して,剰余項を無視すれば次のようなTaylor近似が得られる.
\method[Taylor近似]
{\bf p次Taylor近似}は(\ref{eq22})をみたす微分方程式と$ t_0 < t_1 < t_2 < \cdots < t_n < \cdots $ という間隔 $\Delta_n = t_{n+1} - t_{n}$ の離散化に対して,近似解を
\eq
\label{eq23}
y_{n+1} = y_n + a(t_n,y_n)\Delta_n + \frac{1}{2!}\frac{da}{dt}(t_n,y_n) \Delta_n^2 + \cdots + \frac{1}{p!}\frac{d^{p-1}a}{dt^{p-1}}(t_n)\Delta^p_n
\eqx
によって与える.
\methodx
\ex
例えば,$2$次Taylor近似は,
\eq
y_{n+1} = y_n + a(t_n,y_n)\Delta_n + \frac{1}{2!}\{ a_t + a_x a \} \Delta_n^2
\eqx
$3$次Taylor近似は,
\eq
y_{n+1} = y_n + a(t_n,y_n)\Delta_n + \frac{1}{2!}\{ a_t + a_x a \} \Delta_n^2
+\frac{1}{3!}\{ a_{tt} + 2a_{tx}a + a_{xx}a^2 + a_ta_x +a_x^2 a \} \Delta_n^3
\eqx
によって与えられる.各々の偏微分には$(t_n,y_n)$を代入する.
\exx
しかし,これらのTaylor近似は,与えられた$a$に対して偏微分を計算しなければならないという点において実用的ではない.その場合には,
\eq
a_t(t_n,y_n) \approx \frac{a(t_{n+1},y_n) - a(t_n,y_n)}{\Delta_n},
a_x(t_n,y_n) \approx \frac{a(t_n,y_{n+1}) - a(t_n,y_n)}{y_{n+1} -y_n}
\eqx
という近似を使って偏微分を計算すれば良い.また,$a_x$の計算には$y_{n+1}$が現れるが,
それはEuler法等により予測すれば良い.\par
再び一段法について着目する.一段法はヒューリスティックに
\eq
y_{n+1} = y_n + \Psi(t_n,y_n,\Delta_n) \Delta_n
\eqx
という式の$\Psi(t,y,\Delta)$を計算することが重要であったが,ここではその方法の典型的な例であるRunge-Kutta法について触れる.
\method[2次Runge-Kutta法]
{\bf 2次Runge-Kutta法}は 
\eq
y_{n+1} = y_n + \Psi(t_n,y_n,\Delta_n) \Delta_n,\  \  y_0 = x_0
\eqx
という一段法のincrement functionに対して,
\eq
\Psi(t,y,\Delta) = \alpha \: a(t,x) + \beta \: a(t+\gamma \Delta, x + \gamma a(t,x) \Delta)
\eqx
を代入することによって得られる.
\methodx
ここで,右辺を展開することによって,
\[
\Psi(t,y,\Delta) = (\alpha + \beta)a(t,x) + \gamma \beta(a_t + a_x a )\Delta
+ \frac{1}{2}  \gamma^2 \beta (a_{tt} + 2a_{tx}a + a_{xx}a^2)\Delta^2 + \cdots
\]
が得られる.ここで$3$次Taylor近似と各項を比較すると,
\[
y_{n+1} = y_n + a(t_n,y_n)\Delta_n + \frac{1}{2!}\{ a_t + a_x a \} \Delta_n^2
+\frac{1}{3!}\{ a_{tt} + 2a_{tx}a + a_{xx}a^2 + a_ta_x +a_x^2 a \} \Delta_n^3
\]
\[
\alpha + \beta = 1 ,\ \ \gamma \beta = \frac{1}{2}
\]
が得られる.一方で$\alpha = \beta = 1/2, \gamma =1$を代入するとこれはHeun法(\ref{eq6})となる.
ここで更に$4$次のRunge-Kutta法について紹介する.
\method[4次Runge-Kutta法]
{\bf 4次Runge-Kutta法}は$ t_0 < t_1 < t_2 < \cdots < t_n < \cdots $ という\\
間隔 $\Delta_n = t_{n+1} - t_{n}$ の離散化に対して,近似解を
\eq
y_{n+1} = y_n + \frac{1}{6} \{ k_n^{(1)} +
2k_n^{(2)}+ 2k_n^{(3)} +  k_n^{(4)} \} \Delta_n
\eqx
によって与える.ただし,
\eqa
 k_n^{(1)} &=&  a(t_n,y_n),\\
 k_n^{(2)} &=&  a\left(t_n+ \frac{1}{2} \Delta_n ,y_n + \frac{1}{2} k_n^{(1)} \Delta_n\right),\\
 k_n^{(3)} &=&  a\left(t_n+ \frac{1}{2} \Delta_n ,y_n + \frac{1}{2} k_n^{(2)} \Delta_n\right),\\
 k_n^{(4)} &=&  a\left(t_{n+1} ,y_n + k_n^{(3)} \Delta_n\right),\\
\eqax
\methodx


\end{document}