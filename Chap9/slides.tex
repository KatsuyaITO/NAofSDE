\documentclass[dvipdfmx,cjk]{beamer} 
\AtBeginDvi{\special{pdf:tounicode 90ms-RKSJ-UCS2}} 
\usetheme{CambridgeUS} 
\usecolortheme{seahorse}

\setbeamertemplate{theorems}[numbered]


\numberwithin{equation}{section}
\newtheorem{Thm}     {定理}[section]
\newtheorem{Def}     [Thm]{定義}
\newtheorem{Prop}    [Thm]{命題}
\newtheorem{Fact}    [Thm]{事実}
\newtheorem{Cor}     [Thm]{系}  
\newtheorem{Conj}    [Thm]{予想}
\newtheorem{Ex}      [Thm]{例}  
\newtheorem{Achiom}   [Thm]{公理}
\newtheorem{Method}[Thm]{方法} 
\newtheorem{Rem}  [Thm]{注意}
\newtheorem{Notation}[Thm]{記法}
\newtheorem{Symbol}  [Thm]{記号}
\newtheorem{Prob}    [Thm]{問題}
\makeatletter


\newenvironment{lightgrayleftbar}{%
  \def\FrameCommand{\textcolor{lightgray}{\vrule width 1zw} \hspace{10pt}}% 
  \MakeFramed {\advance\hsize-\width \FrameRestore}}%
{\endMakeFramed}
\def\C{\mathbb C}
\def\N{\mathbb N}
\def\Z{\mathbb Z}
\def\iff{\Leftrightarrow}
\def\R{\mathbb R}
\def\F{\mathcal F}
\def\method{\begin{Method}}
\def\methodx{\end{Method}}
\def\thm{\begin{Thm}}
\def\thmx{\end{Thm}}
\def\prop{\begin{Prop}}
\def\propx{\end{Prop}}
\def\frameb{\begin{frame}}
\def\framex{\end{frame}}
\def\defb{\begin{Def}}
\def\defe{\end{Def}}
\def\defx{\end{Def}}
\newcommand{\supp}{\mathop{\mathrm{supp}}\nolimits}
\def\rem{\begin{Rem}}
\def\remx{\end{Rem}}
\def\prob{\begin{Prob}}
\def\probx{\end{Prob}}
\def\lem{\begin{Lemma}}
\def\lemx{\end{Lemma}}
\def\ex{\begin{Ex}}
\def\exx{\end{Ex}}
\def\cor{\begin{Cor}}
\def\corx{\end{Cor}}
\def\proof{\begin{Proof}}
\def\proofx{\end{Proof}}
\def\eq{\begin{equation}}
\def\eqx{\end{equation}}
\def\eqa{\begin{eqnarray}}
\def\eqax{\end{eqnarray}}
\def\bbL{\mathbb L}
\def\calL{\mathcal L}

    
        
    
    \usepackage[T1]{fontenc}
    % Nicer default font than Computer Modern for most use cases
    \usepackage{palatino}

    % Basic figure setup, for now with no caption control since it's done
    % automatically by Pandoc (which extracts ![](path) syntax from Markdown).
    \usepackage{graphicx}
    % We will generate all images so they have a width \maxwidth. This means
    % that they will get their normal width if they fit onto the page, but
    % are scaled down if they would overflow the margins.
    \makeatletter
    \def\maxwidth{\ifdim\Gin@nat@width>\linewidth\linewidth
    \else\Gin@nat@width\fi}
    \makeatother
    \let\Oldincludegraphics\includegraphics
    % Set max figure width to be 80% of text width, for now hardcoded.
    \renewcommand{\includegraphics}[1]{\Oldincludegraphics[width=.8\maxwidth]{#1}}
    % Ensure that by default, figures have no caption (until we provide a
    % proper Figure object with a Caption API and a way to capture that
    % in the conversion process - todo).
    \usepackage{caption}
    \DeclareCaptionLabelFormat{nolabel}{}
    \captionsetup{labelformat=nolabel}

    \usepackage{adjustbox} % Used to constrain images to a maximum size 
    \usepackage{xcolor} % Allow colors to be defined
    \usepackage{enumerate} % Needed for markdown enumerations to work
    \usepackage{geometry} % Used to adjust the document margins
    \usepackage{amsmath} % Equations
    \usepackage{amssymb} % Equations
    \usepackage{textcomp} % defines textquotesingle
    % Hack from http://tex.stackexchange.com/a/47451/13684:
    \AtBeginDocument{%
        \def\PYZsq{\textquotesingle}% Upright quotes in Pygmentized code
    }
    \usepackage{upquote} % Upright quotes for verbatim code
    \usepackage{eurosym} % defines \euro
    \usepackage[mathletters]{ucs} % Extended unicode (utf-8) support
    \usepackage[utf8x]{inputenc} % Allow utf-8 characters in the tex document
    \usepackage{fancyvrb} % verbatim replacement that allows latex
    \usepackage{grffile} % extends the file name processing of package graphics 
                         % to support a larger range 
    % The hyperref package gives us a pdf with properly built
    % internal navigation ('pdf bookmarks' for the table of contents,
    % internal cross-reference links, web links for URLs, etc.)
    \usepackage{hyperref}
    \usepackage{longtable} % longtable support required by pandoc >1.10
    \usepackage{booktabs}  % table support for pandoc > 1.12.2
    \usepackage[normalem]{ulem} % ulem is needed to support strikethroughs (\sout)
                                % normalem makes italics be italics, not underlines
    

    \usepackage{tikz} % Needed to box output/input
    \usepackage{scrextend} % Used to indent output
    \usepackage{needspace} % Make prompts follow contents
    \usepackage{framed} % Used to draw output that spans multiple pages


    
    
    
    % Colors for the hyperref package
    \definecolor{urlcolor}{rgb}{0,.145,.698}
    \definecolor{linkcolor}{rgb}{.71,0.21,0.01}
    \definecolor{citecolor}{rgb}{.12,.54,.11}

    % ANSI colors
    \definecolor{ansi-black}{HTML}{3E424D}
    \definecolor{ansi-black-intense}{HTML}{282C36}
    \definecolor{ansi-red}{HTML}{E75C58}
    \definecolor{ansi-red-intense}{HTML}{B22B31}
    \definecolor{ansi-green}{HTML}{00A250}
    \definecolor{ansi-green-intense}{HTML}{007427}
    \definecolor{ansi-yellow}{HTML}{DDB62B}
    \definecolor{ansi-yellow-intense}{HTML}{B27D12}
    \definecolor{ansi-blue}{HTML}{208FFB}
    \definecolor{ansi-blue-intense}{HTML}{0065CA}
    \definecolor{ansi-magenta}{HTML}{D160C4}
    \definecolor{ansi-magenta-intense}{HTML}{A03196}
    \definecolor{ansi-cyan}{HTML}{60C6C8}
    \definecolor{ansi-cyan-intense}{HTML}{258F8F}
    \definecolor{ansi-white}{HTML}{C5C1B4}
    \definecolor{ansi-white-intense}{HTML}{A1A6B2}

    % commands and environments needed by pandoc snippets
    % extracted from the output of `pandoc -s`
    \providecommand{\tightlist}{%
      \setlength{\itemsep}{0pt}\setlength{\parskip}{0pt}}
    \DefineVerbatimEnvironment{Highlighting}{Verbatim}{commandchars=\\\{\}}
    % Add ',fontsize=\small' for more characters per line
    \newenvironment{Shaded}{}{}
    \newcommand{\KeywordTok}[1]{\textcolor[rgb]{0.00,0.44,0.13}{\textbf{{#1}}}}
    \newcommand{\DataTypeTok}[1]{\textcolor[rgb]{0.56,0.13,0.00}{{#1}}}
    \newcommand{\DecValTok}[1]{\textcolor[rgb]{0.25,0.63,0.44}{{#1}}}
    \newcommand{\BaseNTok}[1]{\textcolor[rgb]{0.25,0.63,0.44}{{#1}}}
    \newcommand{\FloatTok}[1]{\textcolor[rgb]{0.25,0.63,0.44}{{#1}}}
    \newcommand{\CharTok}[1]{\textcolor[rgb]{0.25,0.44,0.63}{{#1}}}
    \newcommand{\StringTok}[1]{\textcolor[rgb]{0.25,0.44,0.63}{{#1}}}
    \newcommand{\CommentTok}[1]{\textcolor[rgb]{0.38,0.63,0.69}{\textit{{#1}}}}
    \newcommand{\OtherTok}[1]{\textcolor[rgb]{0.00,0.44,0.13}{{#1}}}
    \newcommand{\AlertTok}[1]{\textcolor[rgb]{1.00,0.00,0.00}{\textbf{{#1}}}}
    \newcommand{\FunctionTok}[1]{\textcolor[rgb]{0.02,0.16,0.49}{{#1}}}
    \newcommand{\RegionMarkerTok}[1]{{#1}}
    \newcommand{\ErrorTok}[1]{\textcolor[rgb]{1.00,0.00,0.00}{\textbf{{#1}}}}
    \newcommand{\NormalTok}[1]{{#1}}
    
    % Additional commands for more recent versions of Pandoc
    \newcommand{\ConstantTok}[1]{\textcolor[rgb]{0.53,0.00,0.00}{{#1}}}
    \newcommand{\SpecialCharTok}[1]{\textcolor[rgb]{0.25,0.44,0.63}{{#1}}}
    \newcommand{\VerbatimStringTok}[1]{\textcolor[rgb]{0.25,0.44,0.63}{{#1}}}
    \newcommand{\SpecialStringTok}[1]{\textcolor[rgb]{0.73,0.40,0.53}{{#1}}}
    \newcommand{\ImportTok}[1]{{#1}}
    \newcommand{\DocumentationTok}[1]{\textcolor[rgb]{0.73,0.13,0.13}{\textit{{#1}}}}
    \newcommand{\AnnotationTok}[1]{\textcolor[rgb]{0.38,0.63,0.69}{\textbf{\textit{{#1}}}}}
    \newcommand{\CommentVarTok}[1]{\textcolor[rgb]{0.38,0.63,0.69}{\textbf{\textit{{#1}}}}}
    \newcommand{\VariableTok}[1]{\textcolor[rgb]{0.10,0.09,0.49}{{#1}}}
    \newcommand{\ControlFlowTok}[1]{\textcolor[rgb]{0.00,0.44,0.13}{\textbf{{#1}}}}
    \newcommand{\OperatorTok}[1]{\textcolor[rgb]{0.40,0.40,0.40}{{#1}}}
    \newcommand{\BuiltInTok}[1]{{#1}}
    \newcommand{\ExtensionTok}[1]{{#1}}
    \newcommand{\PreprocessorTok}[1]{\textcolor[rgb]{0.74,0.48,0.00}{{#1}}}
    \newcommand{\AttributeTok}[1]{\textcolor[rgb]{0.49,0.56,0.16}{{#1}}}
    \newcommand{\InformationTok}[1]{\textcolor[rgb]{0.38,0.63,0.69}{\textbf{\textit{{#1}}}}}
    \newcommand{\WarningTok}[1]{\textcolor[rgb]{0.38,0.63,0.69}{\textbf{\textit{{#1}}}}}
    
    
    % Define a nice break command that doesn't care if a line doesn't already
    % exist.
    \def\br{\hspace*{\fill} \\* }
    % Math Jax compatability definitions
    \def\gt{>}
    \def\lt{<}
    % Document parameters
    \title{PC-ExerciseChapter9}
    
    
    

    % Pygments definitions
    
\makeatletter
\def\PY@reset{\let\PY@it=\relax \let\PY@bf=\relax%
    \let\PY@ul=\relax \let\PY@tc=\relax%
    \let\PY@bc=\relax \let\PY@ff=\relax}
\def\PY@tok#1{\csname PY@tok@#1\endcsname}
\def\PY@toks#1+{\ifx\relax#1\empty\else%
    \PY@tok{#1}\expandafter\PY@toks\fi}
\def\PY@do#1{\PY@bc{\PY@tc{\PY@ul{%
    \PY@it{\PY@bf{\PY@ff{#1}}}}}}}
\def\PY#1#2{\PY@reset\PY@toks#1+\relax+\PY@do{#2}}

\expandafter\def\csname PY@tok@nn\endcsname{\let\PY@bf=\textbf\def\PY@tc##1{\textcolor[rgb]{0.00,0.00,1.00}{##1}}}
\expandafter\def\csname PY@tok@il\endcsname{\def\PY@tc##1{\textcolor[rgb]{0.40,0.40,0.40}{##1}}}
\expandafter\def\csname PY@tok@nt\endcsname{\let\PY@bf=\textbf\def\PY@tc##1{\textcolor[rgb]{0.00,0.50,0.00}{##1}}}
\expandafter\def\csname PY@tok@ne\endcsname{\let\PY@bf=\textbf\def\PY@tc##1{\textcolor[rgb]{0.82,0.25,0.23}{##1}}}
\expandafter\def\csname PY@tok@gr\endcsname{\def\PY@tc##1{\textcolor[rgb]{1.00,0.00,0.00}{##1}}}
\expandafter\def\csname PY@tok@s1\endcsname{\def\PY@tc##1{\textcolor[rgb]{0.73,0.13,0.13}{##1}}}
\expandafter\def\csname PY@tok@ss\endcsname{\def\PY@tc##1{\textcolor[rgb]{0.10,0.09,0.49}{##1}}}
\expandafter\def\csname PY@tok@c1\endcsname{\let\PY@it=\textit\def\PY@tc##1{\textcolor[rgb]{0.25,0.50,0.50}{##1}}}
\expandafter\def\csname PY@tok@w\endcsname{\def\PY@tc##1{\textcolor[rgb]{0.73,0.73,0.73}{##1}}}
\expandafter\def\csname PY@tok@nf\endcsname{\def\PY@tc##1{\textcolor[rgb]{0.00,0.00,1.00}{##1}}}
\expandafter\def\csname PY@tok@sr\endcsname{\def\PY@tc##1{\textcolor[rgb]{0.73,0.40,0.53}{##1}}}
\expandafter\def\csname PY@tok@mi\endcsname{\def\PY@tc##1{\textcolor[rgb]{0.40,0.40,0.40}{##1}}}
\expandafter\def\csname PY@tok@kp\endcsname{\def\PY@tc##1{\textcolor[rgb]{0.00,0.50,0.00}{##1}}}
\expandafter\def\csname PY@tok@kr\endcsname{\let\PY@bf=\textbf\def\PY@tc##1{\textcolor[rgb]{0.00,0.50,0.00}{##1}}}
\expandafter\def\csname PY@tok@ni\endcsname{\let\PY@bf=\textbf\def\PY@tc##1{\textcolor[rgb]{0.60,0.60,0.60}{##1}}}
\expandafter\def\csname PY@tok@no\endcsname{\def\PY@tc##1{\textcolor[rgb]{0.53,0.00,0.00}{##1}}}
\expandafter\def\csname PY@tok@bp\endcsname{\def\PY@tc##1{\textcolor[rgb]{0.00,0.50,0.00}{##1}}}
\expandafter\def\csname PY@tok@na\endcsname{\def\PY@tc##1{\textcolor[rgb]{0.49,0.56,0.16}{##1}}}
\expandafter\def\csname PY@tok@se\endcsname{\let\PY@bf=\textbf\def\PY@tc##1{\textcolor[rgb]{0.73,0.40,0.13}{##1}}}
\expandafter\def\csname PY@tok@o\endcsname{\def\PY@tc##1{\textcolor[rgb]{0.40,0.40,0.40}{##1}}}
\expandafter\def\csname PY@tok@nc\endcsname{\let\PY@bf=\textbf\def\PY@tc##1{\textcolor[rgb]{0.00,0.00,1.00}{##1}}}
\expandafter\def\csname PY@tok@ge\endcsname{\let\PY@it=\textit}
\expandafter\def\csname PY@tok@go\endcsname{\def\PY@tc##1{\textcolor[rgb]{0.53,0.53,0.53}{##1}}}
\expandafter\def\csname PY@tok@s\endcsname{\def\PY@tc##1{\textcolor[rgb]{0.73,0.13,0.13}{##1}}}
\expandafter\def\csname PY@tok@kt\endcsname{\def\PY@tc##1{\textcolor[rgb]{0.69,0.00,0.25}{##1}}}
\expandafter\def\csname PY@tok@kd\endcsname{\let\PY@bf=\textbf\def\PY@tc##1{\textcolor[rgb]{0.00,0.50,0.00}{##1}}}
\expandafter\def\csname PY@tok@m\endcsname{\def\PY@tc##1{\textcolor[rgb]{0.40,0.40,0.40}{##1}}}
\expandafter\def\csname PY@tok@gt\endcsname{\def\PY@tc##1{\textcolor[rgb]{0.00,0.27,0.87}{##1}}}
\expandafter\def\csname PY@tok@c\endcsname{\let\PY@it=\textit\def\PY@tc##1{\textcolor[rgb]{0.25,0.50,0.50}{##1}}}
\expandafter\def\csname PY@tok@mh\endcsname{\def\PY@tc##1{\textcolor[rgb]{0.40,0.40,0.40}{##1}}}
\expandafter\def\csname PY@tok@sh\endcsname{\def\PY@tc##1{\textcolor[rgb]{0.73,0.13,0.13}{##1}}}
\expandafter\def\csname PY@tok@cs\endcsname{\let\PY@it=\textit\def\PY@tc##1{\textcolor[rgb]{0.25,0.50,0.50}{##1}}}
\expandafter\def\csname PY@tok@vc\endcsname{\def\PY@tc##1{\textcolor[rgb]{0.10,0.09,0.49}{##1}}}
\expandafter\def\csname PY@tok@gp\endcsname{\let\PY@bf=\textbf\def\PY@tc##1{\textcolor[rgb]{0.00,0.00,0.50}{##1}}}
\expandafter\def\csname PY@tok@kc\endcsname{\let\PY@bf=\textbf\def\PY@tc##1{\textcolor[rgb]{0.00,0.50,0.00}{##1}}}
\expandafter\def\csname PY@tok@mo\endcsname{\def\PY@tc##1{\textcolor[rgb]{0.40,0.40,0.40}{##1}}}
\expandafter\def\csname PY@tok@cp\endcsname{\def\PY@tc##1{\textcolor[rgb]{0.74,0.48,0.00}{##1}}}
\expandafter\def\csname PY@tok@gs\endcsname{\let\PY@bf=\textbf}
\expandafter\def\csname PY@tok@gi\endcsname{\def\PY@tc##1{\textcolor[rgb]{0.00,0.63,0.00}{##1}}}
\expandafter\def\csname PY@tok@err\endcsname{\def\PY@bc##1{\setlength{\fboxsep}{0pt}\fcolorbox[rgb]{1.00,0.00,0.00}{1,1,1}{\strut ##1}}}
\expandafter\def\csname PY@tok@k\endcsname{\let\PY@bf=\textbf\def\PY@tc##1{\textcolor[rgb]{0.00,0.50,0.00}{##1}}}
\expandafter\def\csname PY@tok@gu\endcsname{\let\PY@bf=\textbf\def\PY@tc##1{\textcolor[rgb]{0.50,0.00,0.50}{##1}}}
\expandafter\def\csname PY@tok@mb\endcsname{\def\PY@tc##1{\textcolor[rgb]{0.40,0.40,0.40}{##1}}}
\expandafter\def\csname PY@tok@nd\endcsname{\def\PY@tc##1{\textcolor[rgb]{0.67,0.13,1.00}{##1}}}
\expandafter\def\csname PY@tok@vi\endcsname{\def\PY@tc##1{\textcolor[rgb]{0.10,0.09,0.49}{##1}}}
\expandafter\def\csname PY@tok@si\endcsname{\let\PY@bf=\textbf\def\PY@tc##1{\textcolor[rgb]{0.73,0.40,0.53}{##1}}}
\expandafter\def\csname PY@tok@nl\endcsname{\def\PY@tc##1{\textcolor[rgb]{0.63,0.63,0.00}{##1}}}
\expandafter\def\csname PY@tok@mf\endcsname{\def\PY@tc##1{\textcolor[rgb]{0.40,0.40,0.40}{##1}}}
\expandafter\def\csname PY@tok@sd\endcsname{\let\PY@it=\textit\def\PY@tc##1{\textcolor[rgb]{0.73,0.13,0.13}{##1}}}
\expandafter\def\csname PY@tok@vg\endcsname{\def\PY@tc##1{\textcolor[rgb]{0.10,0.09,0.49}{##1}}}
\expandafter\def\csname PY@tok@ow\endcsname{\let\PY@bf=\textbf\def\PY@tc##1{\textcolor[rgb]{0.67,0.13,1.00}{##1}}}
\expandafter\def\csname PY@tok@sc\endcsname{\def\PY@tc##1{\textcolor[rgb]{0.73,0.13,0.13}{##1}}}
\expandafter\def\csname PY@tok@sb\endcsname{\def\PY@tc##1{\textcolor[rgb]{0.73,0.13,0.13}{##1}}}
\expandafter\def\csname PY@tok@gh\endcsname{\let\PY@bf=\textbf\def\PY@tc##1{\textcolor[rgb]{0.00,0.00,0.50}{##1}}}
\expandafter\def\csname PY@tok@s2\endcsname{\def\PY@tc##1{\textcolor[rgb]{0.73,0.13,0.13}{##1}}}
\expandafter\def\csname PY@tok@cpf\endcsname{\let\PY@it=\textit\def\PY@tc##1{\textcolor[rgb]{0.25,0.50,0.50}{##1}}}
\expandafter\def\csname PY@tok@kn\endcsname{\let\PY@bf=\textbf\def\PY@tc##1{\textcolor[rgb]{0.00,0.50,0.00}{##1}}}
\expandafter\def\csname PY@tok@sx\endcsname{\def\PY@tc##1{\textcolor[rgb]{0.00,0.50,0.00}{##1}}}
\expandafter\def\csname PY@tok@ch\endcsname{\let\PY@it=\textit\def\PY@tc##1{\textcolor[rgb]{0.25,0.50,0.50}{##1}}}
\expandafter\def\csname PY@tok@gd\endcsname{\def\PY@tc##1{\textcolor[rgb]{0.63,0.00,0.00}{##1}}}
\expandafter\def\csname PY@tok@cm\endcsname{\let\PY@it=\textit\def\PY@tc##1{\textcolor[rgb]{0.25,0.50,0.50}{##1}}}
\expandafter\def\csname PY@tok@nv\endcsname{\def\PY@tc##1{\textcolor[rgb]{0.10,0.09,0.49}{##1}}}
\expandafter\def\csname PY@tok@nb\endcsname{\def\PY@tc##1{\textcolor[rgb]{0.00,0.50,0.00}{##1}}}

\def\PYZbs{\char`\\}
\def\PYZus{\char`\_}
\def\PYZob{\char`\{}
\def\PYZcb{\char`\}}
\def\PYZca{\char`\^}
\def\PYZam{\char`\&}
\def\PYZlt{\char`\<}
\def\PYZgt{\char`\>}
\def\PYZsh{\char`\#}
\def\PYZpc{\char`\%}
\def\PYZdl{\char`\$}
\def\PYZhy{\char`\-}
\def\PYZsq{\char`\'}
\def\PYZdq{\char`\"}
\def\PYZti{\char`\~}
% for compatibility with earlier versions
\def\PYZat{@}
\def\PYZlb{[}
\def\PYZrb{]}
\makeatother


    % Exact colors from NB
    \definecolor{incolor}{rgb}{0.0, 0.0, 0.5}
    \definecolor{outcolor}{rgb}{0.545, 0.0, 0.0}



    % Pygments definitions
    
\makeatletter
\def\PY@reset{\let\PY@it=\relax \let\PY@bf=\relax%
    \let\PY@ul=\relax \let\PY@tc=\relax%
    \let\PY@bc=\relax \let\PY@ff=\relax}
\def\PY@tok#1{\csname PY@tok@#1\endcsname}
\def\PY@toks#1+{\ifx\relax#1\empty\else%
    \PY@tok{#1}\expandafter\PY@toks\fi}
\def\PY@do#1{\PY@bc{\PY@tc{\PY@ul{%
    \PY@it{\PY@bf{\PY@ff{#1}}}}}}}
\def\PY#1#2{\PY@reset\PY@toks#1+\relax+\PY@do{#2}}

\expandafter\def\csname PY@tok@nn\endcsname{\let\PY@bf=\textbf\def\PY@tc##1{\textcolor[rgb]{0.00,0.00,1.00}{##1}}}
\expandafter\def\csname PY@tok@il\endcsname{\def\PY@tc##1{\textcolor[rgb]{0.40,0.40,0.40}{##1}}}
\expandafter\def\csname PY@tok@nt\endcsname{\let\PY@bf=\textbf\def\PY@tc##1{\textcolor[rgb]{0.00,0.50,0.00}{##1}}}
\expandafter\def\csname PY@tok@ne\endcsname{\let\PY@bf=\textbf\def\PY@tc##1{\textcolor[rgb]{0.82,0.25,0.23}{##1}}}
\expandafter\def\csname PY@tok@gr\endcsname{\def\PY@tc##1{\textcolor[rgb]{1.00,0.00,0.00}{##1}}}
\expandafter\def\csname PY@tok@s1\endcsname{\def\PY@tc##1{\textcolor[rgb]{0.73,0.13,0.13}{##1}}}
\expandafter\def\csname PY@tok@ss\endcsname{\def\PY@tc##1{\textcolor[rgb]{0.10,0.09,0.49}{##1}}}
\expandafter\def\csname PY@tok@c1\endcsname{\let\PY@it=\textit\def\PY@tc##1{\textcolor[rgb]{0.25,0.50,0.50}{##1}}}
\expandafter\def\csname PY@tok@w\endcsname{\def\PY@tc##1{\textcolor[rgb]{0.73,0.73,0.73}{##1}}}
\expandafter\def\csname PY@tok@nf\endcsname{\def\PY@tc##1{\textcolor[rgb]{0.00,0.00,1.00}{##1}}}
\expandafter\def\csname PY@tok@sr\endcsname{\def\PY@tc##1{\textcolor[rgb]{0.73,0.40,0.53}{##1}}}
\expandafter\def\csname PY@tok@mi\endcsname{\def\PY@tc##1{\textcolor[rgb]{0.40,0.40,0.40}{##1}}}
\expandafter\def\csname PY@tok@kp\endcsname{\def\PY@tc##1{\textcolor[rgb]{0.00,0.50,0.00}{##1}}}
\expandafter\def\csname PY@tok@kr\endcsname{\let\PY@bf=\textbf\def\PY@tc##1{\textcolor[rgb]{0.00,0.50,0.00}{##1}}}
\expandafter\def\csname PY@tok@ni\endcsname{\let\PY@bf=\textbf\def\PY@tc##1{\textcolor[rgb]{0.60,0.60,0.60}{##1}}}
\expandafter\def\csname PY@tok@no\endcsname{\def\PY@tc##1{\textcolor[rgb]{0.53,0.00,0.00}{##1}}}
\expandafter\def\csname PY@tok@bp\endcsname{\def\PY@tc##1{\textcolor[rgb]{0.00,0.50,0.00}{##1}}}
\expandafter\def\csname PY@tok@na\endcsname{\def\PY@tc##1{\textcolor[rgb]{0.49,0.56,0.16}{##1}}}
\expandafter\def\csname PY@tok@se\endcsname{\let\PY@bf=\textbf\def\PY@tc##1{\textcolor[rgb]{0.73,0.40,0.13}{##1}}}
\expandafter\def\csname PY@tok@o\endcsname{\def\PY@tc##1{\textcolor[rgb]{0.40,0.40,0.40}{##1}}}
\expandafter\def\csname PY@tok@nc\endcsname{\let\PY@bf=\textbf\def\PY@tc##1{\textcolor[rgb]{0.00,0.00,1.00}{##1}}}
\expandafter\def\csname PY@tok@ge\endcsname{\let\PY@it=\textit}
\expandafter\def\csname PY@tok@go\endcsname{\def\PY@tc##1{\textcolor[rgb]{0.53,0.53,0.53}{##1}}}
\expandafter\def\csname PY@tok@s\endcsname{\def\PY@tc##1{\textcolor[rgb]{0.73,0.13,0.13}{##1}}}
\expandafter\def\csname PY@tok@kt\endcsname{\def\PY@tc##1{\textcolor[rgb]{0.69,0.00,0.25}{##1}}}
\expandafter\def\csname PY@tok@kd\endcsname{\let\PY@bf=\textbf\def\PY@tc##1{\textcolor[rgb]{0.00,0.50,0.00}{##1}}}
\expandafter\def\csname PY@tok@m\endcsname{\def\PY@tc##1{\textcolor[rgb]{0.40,0.40,0.40}{##1}}}
\expandafter\def\csname PY@tok@gt\endcsname{\def\PY@tc##1{\textcolor[rgb]{0.00,0.27,0.87}{##1}}}
\expandafter\def\csname PY@tok@c\endcsname{\let\PY@it=\textit\def\PY@tc##1{\textcolor[rgb]{0.25,0.50,0.50}{##1}}}
\expandafter\def\csname PY@tok@mh\endcsname{\def\PY@tc##1{\textcolor[rgb]{0.40,0.40,0.40}{##1}}}
\expandafter\def\csname PY@tok@sh\endcsname{\def\PY@tc##1{\textcolor[rgb]{0.73,0.13,0.13}{##1}}}
\expandafter\def\csname PY@tok@cs\endcsname{\let\PY@it=\textit\def\PY@tc##1{\textcolor[rgb]{0.25,0.50,0.50}{##1}}}
\expandafter\def\csname PY@tok@vc\endcsname{\def\PY@tc##1{\textcolor[rgb]{0.10,0.09,0.49}{##1}}}
\expandafter\def\csname PY@tok@gp\endcsname{\let\PY@bf=\textbf\def\PY@tc##1{\textcolor[rgb]{0.00,0.00,0.50}{##1}}}
\expandafter\def\csname PY@tok@kc\endcsname{\let\PY@bf=\textbf\def\PY@tc##1{\textcolor[rgb]{0.00,0.50,0.00}{##1}}}
\expandafter\def\csname PY@tok@mo\endcsname{\def\PY@tc##1{\textcolor[rgb]{0.40,0.40,0.40}{##1}}}
\expandafter\def\csname PY@tok@cp\endcsname{\def\PY@tc##1{\textcolor[rgb]{0.74,0.48,0.00}{##1}}}
\expandafter\def\csname PY@tok@gs\endcsname{\let\PY@bf=\textbf}
\expandafter\def\csname PY@tok@gi\endcsname{\def\PY@tc##1{\textcolor[rgb]{0.00,0.63,0.00}{##1}}}
\expandafter\def\csname PY@tok@err\endcsname{\def\PY@bc##1{\setlength{\fboxsep}{0pt}\fcolorbox[rgb]{1.00,0.00,0.00}{1,1,1}{\strut ##1}}}
\expandafter\def\csname PY@tok@k\endcsname{\let\PY@bf=\textbf\def\PY@tc##1{\textcolor[rgb]{0.00,0.50,0.00}{##1}}}
\expandafter\def\csname PY@tok@gu\endcsname{\let\PY@bf=\textbf\def\PY@tc##1{\textcolor[rgb]{0.50,0.00,0.50}{##1}}}
\expandafter\def\csname PY@tok@mb\endcsname{\def\PY@tc##1{\textcolor[rgb]{0.40,0.40,0.40}{##1}}}
\expandafter\def\csname PY@tok@nd\endcsname{\def\PY@tc##1{\textcolor[rgb]{0.67,0.13,1.00}{##1}}}
\expandafter\def\csname PY@tok@vi\endcsname{\def\PY@tc##1{\textcolor[rgb]{0.10,0.09,0.49}{##1}}}
\expandafter\def\csname PY@tok@si\endcsname{\let\PY@bf=\textbf\def\PY@tc##1{\textcolor[rgb]{0.73,0.40,0.53}{##1}}}
\expandafter\def\csname PY@tok@nl\endcsname{\def\PY@tc##1{\textcolor[rgb]{0.63,0.63,0.00}{##1}}}
\expandafter\def\csname PY@tok@mf\endcsname{\def\PY@tc##1{\textcolor[rgb]{0.40,0.40,0.40}{##1}}}
\expandafter\def\csname PY@tok@sd\endcsname{\let\PY@it=\textit\def\PY@tc##1{\textcolor[rgb]{0.73,0.13,0.13}{##1}}}
\expandafter\def\csname PY@tok@vg\endcsname{\def\PY@tc##1{\textcolor[rgb]{0.10,0.09,0.49}{##1}}}
\expandafter\def\csname PY@tok@ow\endcsname{\let\PY@bf=\textbf\def\PY@tc##1{\textcolor[rgb]{0.67,0.13,1.00}{##1}}}
\expandafter\def\csname PY@tok@sc\endcsname{\def\PY@tc##1{\textcolor[rgb]{0.73,0.13,0.13}{##1}}}
\expandafter\def\csname PY@tok@sb\endcsname{\def\PY@tc##1{\textcolor[rgb]{0.73,0.13,0.13}{##1}}}
\expandafter\def\csname PY@tok@gh\endcsname{\let\PY@bf=\textbf\def\PY@tc##1{\textcolor[rgb]{0.00,0.00,0.50}{##1}}}
\expandafter\def\csname PY@tok@s2\endcsname{\def\PY@tc##1{\textcolor[rgb]{0.73,0.13,0.13}{##1}}}
\expandafter\def\csname PY@tok@cpf\endcsname{\let\PY@it=\textit\def\PY@tc##1{\textcolor[rgb]{0.25,0.50,0.50}{##1}}}
\expandafter\def\csname PY@tok@kn\endcsname{\let\PY@bf=\textbf\def\PY@tc##1{\textcolor[rgb]{0.00,0.50,0.00}{##1}}}
\expandafter\def\csname PY@tok@sx\endcsname{\def\PY@tc##1{\textcolor[rgb]{0.00,0.50,0.00}{##1}}}
\expandafter\def\csname PY@tok@ch\endcsname{\let\PY@it=\textit\def\PY@tc##1{\textcolor[rgb]{0.25,0.50,0.50}{##1}}}
\expandafter\def\csname PY@tok@gd\endcsname{\def\PY@tc##1{\textcolor[rgb]{0.63,0.00,0.00}{##1}}}
\expandafter\def\csname PY@tok@cm\endcsname{\let\PY@it=\textit\def\PY@tc##1{\textcolor[rgb]{0.25,0.50,0.50}{##1}}}
\expandafter\def\csname PY@tok@nv\endcsname{\def\PY@tc##1{\textcolor[rgb]{0.10,0.09,0.49}{##1}}}
\expandafter\def\csname PY@tok@nb\endcsname{\def\PY@tc##1{\textcolor[rgb]{0.00,0.50,0.00}{##1}}}

\def\PYZbs{\char`\\}
\def\PYZus{\char`\_}
\def\PYZob{\char`\{}
\def\PYZcb{\char`\}}
\def\PYZca{\char`\^}
\def\PYZam{\char`\&}
\def\PYZlt{\char`\<}
\def\PYZgt{\char`\>}
\def\PYZsh{\char`\#}
\def\PYZpc{\char`\%}
\def\PYZdl{\char`\$}
\def\PYZhy{\char`\-}
\def\PYZsq{\char`\'}
\def\PYZdq{\char`\"}
\def\PYZti{\char`\~}
% for compatibility with earlier versions
\def\PYZat{@}
\def\PYZlb{[}
\def\PYZrb{]}
\makeatother


    % NB prompt colors
    \definecolor{nbframe-border}{rgb}{0.867,0.867,0.867}
    \definecolor{nbframe-bg}{rgb}{0.969,0.969,0.969}
    \definecolor{nbframe-in-prompt}{rgb}{0.0,0.0,0.502}
    \definecolor{nbframe-out-prompt}{rgb}{0.545,0.0,0.0}

    % NB prompt lengths
    \newlength{\inputpadding}
    \setlength{\inputpadding}{0.5em}
    \newlength{\cellleftmargin}
    \setlength{\cellleftmargin}{0.15\linewidth}
    \newlength{\borderthickness}
    \setlength{\borderthickness}{0.4pt}
    \newlength{\smallerfontscale}
    \setlength{\smallerfontscale}{9.5pt}

    % NB prompt font size
    \def\smaller{\fontsize{\smallerfontscale}{\smallerfontscale}\selectfont}

    % Define a background layer, in which the nb prompt shape is drawn
    \pgfdeclarelayer{background}
    \pgfsetlayers{background,main}
    \usetikzlibrary{calc}

    % define styles for the normal border and the torn border
    \tikzset{
      normal border/.style={draw=nbframe-border, fill=nbframe-bg,
        rectangle, rounded corners=2.5pt, line width=\borderthickness},
      torn border/.style={draw=white, fill=white, line width=\borderthickness}}

    % Macro to draw the shape behind the text, when it fits completly in the
    % page
    \def\notebookcellframe#1{%
    \tikz{%
      \node[inner sep=\inputpadding] (A) {#1};% Draw the text of the node
      \begin{pgfonlayer}{background}% Draw the shape behind
      \fill[normal border]%
            (A.south east) -- ($(A.south west)+(\cellleftmargin,0)$) -- 
            ($(A.north west)+(\cellleftmargin,0)$) -- (A.north east) -- cycle;
      \end{pgfonlayer}}}%

    % Macro to draw the shape, when the text will continue in next page
    \def\notebookcellframetop#1{%
    \tikz{%
      \node[inner sep=\inputpadding] (A) {#1};    % Draw the text of the node
      \begin{pgfonlayer}{background}    
      \fill[normal border]              % Draw the ``complete shape'' behind
            (A.south east) -- ($(A.south west)+(\cellleftmargin,0)$) -- 
            ($(A.north west)+(\cellleftmargin,0)$) -- (A.north east) -- cycle;
      \fill[torn border]                % Add the torn lower border
            ($(A.south east)-(0,.1)$) -- ($(A.south west)+(\cellleftmargin,-.1)$) -- 
            ($(A.south west)+(\cellleftmargin,.1)$) -- ($(A.south east)+(0,.1)$) -- cycle;
      \end{pgfonlayer}}}

    % Macro to draw the shape, when the text continues from previous page
    \def\notebookcellframebottom#1{%
    \tikz{%
      \node[inner sep=\inputpadding] (A) {#1};   % Draw the text of the node
      \begin{pgfonlayer}{background}   
      \fill[normal border]             % Draw the ``complete shape'' behind
            (A.south east) -- ($(A.south west)+(\cellleftmargin,0)$) -- 
            ($(A.north west)+(\cellleftmargin,0)$) -- (A.north east) -- cycle;
      \fill[torn border]               % Add the torn upper border
            ($(A.north east)-(0,.1)$) -- ($(A.north west)+(\cellleftmargin,-.1)$) -- 
            ($(A.north west)+(\cellleftmargin,.1)$) -- ($(A.north east)+(0,.1)$) -- cycle;
      \end{pgfonlayer}}}

    % Macro to draw the shape, when both the text continues from previous page
    % and it will continue in next page
    \def\notebookcellframemiddle#1{%
    \tikz{%
      \node[inner sep=\inputpadding] (A) {#1};   % Draw the text of the node
      \begin{pgfonlayer}{background}   
      \fill[normal border]             % Draw the ``complete shape'' behind
            (A.south east) -- ($(A.south west)+(\cellleftmargin,0)$) -- 
            ($(A.north west)+(\cellleftmargin,0)$) -- (A.north east) -- cycle;
      \fill[torn border]               % Add the torn lower border
            ($(A.south east)-(0,.1)$) -- ($(A.south west)+(\cellleftmargin,-.1)$) -- 
            ($(A.south west)+(\cellleftmargin,.1)$) -- ($(A.south east)+(0,.1)$) -- cycle;
      \fill[torn border]               % Add the torn upper border
            ($(A.north east)-(0,.1)$) -- ($(A.north west)+(\cellleftmargin,-.1)$) -- 
            ($(A.north west)+(\cellleftmargin,.1)$) -- ($(A.north east)+(0,.1)$) -- cycle;
      \end{pgfonlayer}}}

    % Define the environment which puts the frame
    % In this case, the environment also accepts an argument with an optional
    % title (which defaults to ``Example'', which is typeset in a box overlaid
    % on the top border
    \newenvironment{notebookcell}[1][0]{%
      \def\FrameCommand{\notebookcellframe}%
      \def\FirstFrameCommand{\notebookcellframetop}%
      \def\LastFrameCommand{\notebookcellframebottom}%
      \def\MidFrameCommand{\notebookcellframemiddle}%
      \par\vspace{1\baselineskip}%
      \MakeFramed {\FrameRestore}%
      \noindent\tikz\node[inner sep=0em] at ($(A.north west)-(0,0)$) {%
      \begin{minipage}{\cellleftmargin}%
    \hfill%
    {\smaller%
    \tt%
    \color{nbframe-in-prompt}%
    In[#1]:}%
    \hspace{\inputpadding}%
    \hspace{2pt}%
    \hspace{3pt}%
    \end{minipage}%%
      }; \par}%
    {\endMakeFramed}



    
    % Prevent overflowing lines due to hard-to-break entities
    \sloppy 
    % Setup hyperref package
    \hypersetup{
      breaklinks=true,  % so long urls are correctly broken across lines
      colorlinks=true,
      urlcolor=urlcolor,
      linkcolor=linkcolor,
      citecolor=citecolor,
      }
    % Slightly bigger margins than the latex defaults
   
\usepackage{url}

\begin{document}
\title[Chap8]{Chapter.9 確率微分方程式の離散時間近似} 
\author[Katsuya ITO]{伊藤克哉} 
\institute[UT]{東京大学}
\date{2016/10/03}

\begin{frame}
\titlepage    
\end{frame}
\begin{frame}
講義ノート・スライド・コードは\\
\url{https://github.com/KatsuyaITO/NSofSDE}
\end{frame}
\begin{frame}
\tableofcontents
\end{frame}


\section{確率論からの入門}
\begin{frame}{情報増大系と適合}

以下では$(\Omega,\F,P)$という確率空間を考える.
\defb
$(\F_t)_{t\ge0}$が{\bf 情報増大系}であるとは,
\[
\F_t \subset \F : \mbox{sub}\sigma\mbox{-algかつ}\  0\le s \le t \Rightarrow \F_s \subset \F_s
\]
となることである.\\
また$d$次元確率過程$X=(X_t)_{t\ge 0}$が情報増大系$(\F_t)_{t\ge0}$に対して{\bf 適合(adapted)}しているとは,
$\forall t \ge 0 \ : \ X_t:\Omega \to \R^d$ が$\F_t$可測であるということである.
\defx

\end{frame}

\begin{frame}{Brown運動}
\defb
確率過程$B=(B_t)_{t\ge 0}$が実数値Brown運動であるとは,次を満たすことである.\\
(i) \ $B_0 = 0 \  $a.s.\\
(ii)\ $\forall \omega \in \Omega:B_t(\omega)$は連続である.\\
(iii)\ $0 < t_0 < t_1 < \cdots < t_n $という任意の細分に対して,$\{B_{t_j} - B_{t_{j-1}}\}_{i}$は互いに独立で
$N(0,t_i-t_{i-1})$に従う.
\defx
\end{frame}

\begin{frame}{マルチンゲール}
\defb
右連続な確率過程$X=(X_t)_{t\ge 0}$が$(\F_t)_{t\ge 0}$-マルチンゲールであるとは,次を満たすことである.\\
(i)\ $\forall t \ge 0:\ E[|X_t|] < \infty$\\
(ii)\ $X$は$(\F_t)_t$適合である.\\
(iii)\ $0\le \forall s \le t:\ E[X_t|\F_s] = X_s\ $a.s.である.
\defx

\end{frame}

\begin{frame}{伊藤積分}
\defb[伊藤積分]
次のようにして確率過程の族を表す.\\
\[
\bbL_T^2 := \{f:\mbox{確率過程}\  | f_t(\omega)\mbox{は可測で}||f||_{\bbL_T^2}:= E[\int_{0}^T f_t^2 dt] < \infty \}
\]
\[
\calL_T^2 = \calL_T^2 (\F_t) := \{ f \in \bbL_T^2 | f\mbox{は}(\F_t)\mbox{適合}\}
\]
\defx
\end{frame}

\begin{frame}{伊藤積分の続き}
$f=(f_t)$は次のように表される階段過程であるとする.\\
\[
f_t = \sum_{j=1}^n \tilde{f_j} 1_{[t_{j-1},t_j)}(t) ,\ t\in [0,T]
\]
(ただし,$\tilde{f_j}$は$\F_{t_{j-1}}$可測で有界な確率変数)\\
このとき, 確率過程$f$の確率積分を,
\[
M_t(f) \equiv \int_0^t f_s dB_s := \sum_{j=1}^n \tilde{f_j} (B_{t\wedge t_j} - B_{t\wedge t_{j-1}})
\]
として定める.
\end{frame}


\begin{frame}
また一般の可測な$(\F_t)-$適合な確率過程$f$に対して,$f^n$という階段過程の列が存在して,$||f-f^{(n)}||_{\bbL_T^2} \to 0 ,\ n\to \infty$
とできる.\\
故に,$f$の確率積分を
\[
\int_0^t f_sdB_s := \lim_{n\to\infty} M(f^{(n)})
\]
によって定める.
これは$\mathcal M _T$の元として$f^{(n)}$のとり方によらず一意的に定まる.\\
そしてこれを$[0,\infty)$に拡張して確率積分を定義することもできる.\\
また同様にして次の確率過程の族を定める.\\
\[
\calL^2(\F_t) := \{f=(f_t)_{t\ge 0}:\mbox{確率過程}\  | \forall T>0 : \ (f_t)_{t\in [0,T]} \in \calL^2_T \}
\]


\end{frame}


\begin{frame}{伊藤積分の性質}
\thm
確率積分$M_t(f) = \int_0^t f_sdB_s$は次をみたす.\\
(1)\[M_t(f)\mbox{は} \F_t \mbox{マルチンゲールである.}\]
(2)\[E(M_t(f))=0\]
(3)\[E(M_t(f)^2) = \int_0^T E(f(t,-)^2)dt\]
(4)\[M_t(af+bg) = aM_t(f)+bM_t(g)\ a.s.\]
\thmx
\end{frame}


\begin{frame}{確率微分方程式}
\defb
確率過程$X = (X_t)_{t\ge 0}$が次を満たすとき,{\bf 確率微分方程式(stochastic differential equation)}
\eq
dX_t = a(t,X_t)dt + b(t,X_t)dW_t \ \ X_{t_0} =X_0
\eqx
の解であるという.\\
$X$は$(\Omega,\F,P)$上で定義された$\F_t$適合かつ可測な$\R$値連続確率過程で,\\
(i)
\[
  a(t,X_t) \in L^1_{loc} ([0,\infty)),\  b(t,X_t) \in L^2(\F_t)\  a.s.
\]
(ii)確率積分方程式
\[
X_t = X_0 + \int_0^t a(s,X_s)ds +  \int_0^t b(s,X_s)dB_s
\]
を満たす.
\defx
\end{frame}

\begin{frame}{解の一意性}
\defb
$[t_0,T]$上の確率微分方程式が{\bf pathwise unique}な解を持つとは,\\
任意の2組の解$X_t,\tilde{X}_t$が,
\[
P \big( \sup_{t_0\le t\le T} \big| X_t - \tilde{X}_t \big | > 0 \big) = 0
\]
を満たすということである.
\defx
\end{frame}


\begin{frame}
\thm[解の一意性]
\label{thm_uniq}
$[t_0,T]$上の確率微分方程式
\[
dX_t = a(t,X_t)dt + b(t,X_t)dW_t \ ,\ X_{t_0} =X_0
\]
は次の$4$条件を満たすとき
\[
\sup_{t_0\le t\le T} E(|X_t|^2) < \infty
\]
を満たすようなpathwise unique な強解$X_t$を$[t_0,T]$上持つ\\

\thmx
\end{frame}

\begin{frame}
$(A1)$ (可測性) $a(t,x),b(t,x)$は$[t_0,T] \times\R$で$L^2$可測.\\
$(A2)$(Lipschitz条件)次を満たす定数$K>0$が存在する.
\[
|a(t,x) - a(t,y) | \le K |x-y|
\]
\[
|b(t,x) - b(t,y) | \le K |x-y|
\]
$(A3)$ 次を満たす定数$K>0$が存在する.
\[
|a(t,x)|^2 \le K^2 (1+|x|^2)
\]
\[
|b(t,x)|^2 \le K^2 (1+|x|^2)
\]
$(A4)$
$X_{t_0}$は$\F_{t_0}$可測で$E(|X_{t_0}|^2)<\infty$を満たす.
\end{frame}


\begin{frame}

\thm[伊藤の公式]
$U:[0,T]\times\R \to \R$は$C^2$級であるとする.\\
$X_t$は次の確率微分方程式の解であるとする.
\[
dX_t = e(t,\omega)dt + f(t,\omega)dW_t(\omega)
\]
このとき,$Y_t=U(t,X_t)$は
\[
Y_t - Y_s = \int_s^t \{ \frac{\partial U}{\partial t}(u,X_u)
+ e_u\frac{\partial U}{\partial x}(u,X_u)
+ \frac{1}{2} f_u^2 \frac{\partial^2 U}{\partial x^2}(u,X_u)\} du
\]
\[
+ \int_s^t f_u \frac{\partial U}{\partial x}(u,X_u)dW_u
\]
をalmost surelyに満たす
\thmx
\end{frame}
\section{確率微分方程式の数値解析}

\section{近似解の実装}
\begin{frame}[containsverbatim]
\begin{Verbatim}[commandchars=\\\{\}]

\PY{k}{class} \PY{n+nc}{Process}\PY{p}{:}
    \PY{k}{def} \PY{n+nf}{show}\PY{p}{(}\PY{n+nb+bp}{self}\PY{p}{)}\PY{p}{:}
        \PY{n}{plt}\PY{o}{.}\PY{n}{plot}\PY{p}{(}\PY{n+nb+bp}{self}\PY{o}{.}\PY{n}{show\PYZus{}x}\PY{p}{,}\PY{n+nb+bp}{self}\PY{o}{.}\PY{n}{show\PYZus{}y}\PY{p}{)}
    \PY{k}{def} \PY{n+nf}{at}\PY{p}{(}\PY{n+nb+bp}{self}\PY{p}{,}\PY{n}{t}\PY{p}{)}\PY{p}{:}
        \PY{k}{if} \PY{n}{t} \PY{o+ow}{in} \PY{n+nb+bp}{self}\PY{o}{.}\PY{n}{dic}\PY{p}{:}
            \PY{k}{return} \PY{n+nb+bp}{self}\PY{o}{.}\PY{n}{dic}\PY{p}{[}\PY{n}{t}\PY{p}{]}
        \PY{k}{else}\PY{p}{:}
            \PY{k}{return} \PY{n+nb+bp}{self}\PY{o}{.}\PY{n}{show\PYZus{}y}\PY{p}{[}
                \PY{n}{np}\PY{o}{.}\PY{n}{searchsorted}\PY{p}{(}\PY{n+nb+bp}{self}\PY{o}{.}\PY{n}{show\PYZus{}x}\PY{p}{,} \PY{n}{t}\PY{p}{)}\PY{p}{]}



\end{Verbatim}
\end{frame}

\begin{frame}[containsverbatim]
\begin{Verbatim}[commandchars=\\\{\}]

\PY{k}{class} \PY{n+nc}{Wiener}\PY{p}{(}\PY{n}{Process}\PY{p}{)}\PY{p}{:}
    \PY{k}{def} \PY{n+nf}{\PYZus{}\PYZus{}init\PYZus{}\PYZus{}}\PY{p}{(}\PY{n+nb+bp}{self}\PY{p}{,} \PY{n}{start}\PY{p}{,} \PY{n}{end}\PY{p}{,}\PY{n}{n}\PY{p}{)}\PY{p}{:}
        \PY{n}{t} \PY{o}{=} \PY{n+nb}{list}\PY{p}{(}\PY{n}{np}\PY{o}{.}\PY{n}{linspace}\PY{p}{(}\PY{n}{start}\PY{p}{,}\PY{n}{end}\PY{p}{,}\PY{n}{n}\PY{o}{+}\PY{l+m+mi}{1}\PY{p}{)}\PY{p}{)}
        \PY{n}{w} \PY{o}{=}\PY{p}{[}\PY{l+m+mi}{0}\PY{p}{]}
        \PY{n}{sigma} \PY{o}{=} \PY{n}{math}\PY{o}{.}\PY{n}{sqrt}\PY{p}{(}\PY{n}{t}\PY{p}{[}\PY{l+m+mi}{1}\PY{p}{]}\PY{o}{\PYZhy{}}\PY{n}{t}\PY{p}{[}\PY{l+m+mi}{0}\PY{p}{]}\PY{p}{)}
        \PY{k}{for} \PY{n}{tn} \PY{o+ow}{in} \PY{n}{t}\PY{p}{[}\PY{p}{:}\PY{o}{\PYZhy{}}\PY{l+m+mi}{1}\PY{p}{]}\PY{p}{:}
            \PY{n}{w}\PY{o}{.}\PY{n}{append}\PY{p}{(}\PY{n}{w}\PY{p}{[}\PY{o}{\PYZhy{}}\PY{l+m+mi}{1}\PY{p}{]}\PY{o}{+}
                     \PY{n}{np}\PY{o}{.}\PY{n}{random}\PY{o}{.}\PY{n}{normal}\PY{p}{(}\PY{l+m+mi}{0}\PY{p}{,}\PY{n}{sigma}\PY{p}{)}\PY{p}{)}
        \PY{n+nb+bp}{self}\PY{o}{.}\PY{n}{start} \PY{o}{=} \PY{n}{start}
        \PY{n+nb+bp}{self}\PY{o}{.}\PY{n}{end} \PY{o}{=} \PY{n}{end}
        \PY{n+nb+bp}{self}\PY{o}{.}\PY{n}{t} \PY{o}{=} \PY{n}{t}
        \PY{n+nb+bp}{self}\PY{o}{.}\PY{n}{w} \PY{o}{=} \PY{n}{w}
        \PY{n+nb+bp}{self}\PY{o}{.}\PY{n}{dic} \PY{o}{=}  \PY{n+nb}{dict}\PY{p}{(}\PY{n+nb}{zip}\PY{p}{(}\PY{n}{t}\PY{p}{,}\PY{n}{w}\PY{p}{)}\PY{p}{)}
        \PY{n+nb+bp}{self}\PY{o}{.}\PY{n}{show\PYZus{}x} \PY{o}{=}  \PY{n}{t}
        \PY{n+nb+bp}{self}\PY{o}{.}\PY{n}{show\PYZus{}y} \PY{o}{=} \PY{n}{w}
        
\end{Verbatim}

\end{frame}

\begin{frame}[containsverbatim]
\begin{Verbatim}[commandchars=\\\{\}]
\PY{k}{class} \PY{n+nc}{Solution}\PY{p}{(}\PY{n}{Process}\PY{p}{)}\PY{p}{:}
    \PY{k}{def} \PY{n+nf}{\PYZus{}\PYZus{}init\PYZus{}\PYZus{}}\PY{p}{(}\PY{n+nb+bp}{self}\PY{p}{,}\PY{n}{t}\PY{p}{,}\PY{n}{f}\PY{p}{,}\PY{n}{w}\PY{p}{)}\PY{p}{:}
        \PY{n+nb+bp}{self}\PY{o}{.}\PY{n}{x} \PY{o}{=} \PY{p}{[}\PY{n}{f}\PY{p}{(}\PY{n}{tn}\PY{p}{,}\PY{n}{w}\PY{p}{)} \PY{k}{for} \PY{n}{tn} \PY{o+ow}{in} \PY{n}{t}\PY{p}{]}
        \PY{n+nb+bp}{self}\PY{o}{.}\PY{n}{f} \PY{o}{=} \PY{n}{f}
        \PY{n+nb+bp}{self}\PY{o}{.}\PY{n}{w} \PY{o}{=} \PY{n}{w}
        \PY{n+nb+bp}{self}\PY{o}{.}\PY{n}{t} \PY{o}{=} \PY{n}{t}
        \PY{n+nb+bp}{self}\PY{o}{.}\PY{n}{dic} \PY{o}{=} \PY{n+nb}{dict}\PY{p}{(}\PY{n+nb}{zip}\PY{p}{(}\PY{n}{t}\PY{p}{,}\PY{n+nb+bp}{self}\PY{o}{.}\PY{n}{x}\PY{p}{)}\PY{p}{)}
        \PY{n+nb+bp}{self}\PY{o}{.}\PY{n}{show\PYZus{}x} \PY{o}{=}  \PY{n}{t}
        \PY{n+nb+bp}{self}\PY{o}{.}\PY{n}{show\PYZus{}y} \PY{o}{=} \PY{n+nb+bp}{self}\PY{o}{.}\PY{n}{x}

\end{Verbatim}
\end{frame}


\begin{frame}[containsverbatim]
\begin{Verbatim}[commandchars=\\\{\}]
        
\PY{k}{class} \PY{n+nc}{Euler\PYZus{}Maruyama}\PY{p}{(}\PY{n}{Process}\PY{p}{)}\PY{p}{:}
    \PY{k}{def} \PY{n+nf}{\PYZus{}\PYZus{}init\PYZus{}\PYZus{}}\PY{p}{(}\PY{n+nb+bp}{self}\PY{p}{,}\PY{n}{a}\PY{p}{,}\PY{n}{b}\PY{p}{,}\PY{n}{x0}\PY{p}{,}\PY{n}{w}\PY{p}{,}\PY{n}{delta}\PY{p}{)}\PY{p}{:}
        \PY{n}{t} \PY{o}{=} \PY{n+nb}{list}\PY{p}{(}\PY{n}{np}\PY{o}{.}\PY{n}{arange}\PY{p}{(}\PY{n}{w}\PY{o}{.}\PY{n}{start}\PY{p}{,}\PY{n}{w}\PY{o}{.}\PY{n}{end}\PY{o}{+}\PY{n}{delta}\PY{p}{,}\PY{n}{delta}\PY{p}{)}\PY{p}{)}
        \PY{n}{y} \PY{o}{=}\PY{p}{[}\PY{n}{x0}\PY{p}{]}
        \PY{n}{l} \PY{o}{=} \PY{n+nb}{len}\PY{p}{(}\PY{n}{t}\PY{p}{)}
        
        \PY{k}{for} \PY{n}{i} \PY{o+ow}{in} \PY{n+nb}{range}\PY{p}{(}\PY{n}{l}\PY{o}{\PYZhy{}}\PY{l+m+mi}{1}\PY{p}{)}\PY{p}{:}
            \PY{n}{y}\PY{o}{.}\PY{n}{append}\PY{p}{(}\PY{n}{y}\PY{p}{[}\PY{o}{\PYZhy{}}\PY{l+m+mi}{1}\PY{p}{]}
                     \PY{o}{+}\PY{n}{a}\PY{p}{(}\PY{n}{t}\PY{p}{[}\PY{n}{i}\PY{p}{]}\PY{p}{,}\PY{n}{y}\PY{p}{[}\PY{o}{\PYZhy{}}\PY{l+m+mi}{1}\PY{p}{]}\PY{p}{)}\PY{o}{*}\PY{p}{(}\PY{n}{t}\PY{p}{[}\PY{n}{i}\PY{o}{+}\PY{l+m+mi}{1}\PY{p}{]}\PY{o}{\PYZhy{}}\PY{n}{t}\PY{p}{[}\PY{n}{i}\PY{p}{]}\PY{p}{)}
                     \PY{o}{+}\PY{n}{b}\PY{p}{(}\PY{n}{t}\PY{p}{[}\PY{n}{i}\PY{p}{]}\PY{p}{,}\PY{n}{y}\PY{p}{[}\PY{o}{\PYZhy{}}\PY{l+m+mi}{1}\PY{p}{]}\PY{p}{)}\PY{o}{*}\PY{p}{(}\PY{n}{w}\PY{o}{.}\PY{n}{at}\PY{p}{(}\PY{n}{t}\PY{p}{[}\PY{n}{i}\PY{o}{+}\PY{l+m+mi}{1}\PY{p}{]}\PY{p}{)}\PY{o}{\PYZhy{}}\PY{n}{w}\PY{o}{.}\PY{n}{at}\PY{p}{(}\PY{n}{t}\PY{p}{[}\PY{n}{i}\PY{p}{]}\PY{p}{)}\PY{p}{)}\PY{p}{)}

        \PY{n+nb+bp}{self}\PY{o}{.}\PY{n}{t} \PY{o}{=} \PY{n}{t}
        \PY{n+nb+bp}{self}\PY{o}{.}\PY{n}{w} \PY{o}{=} \PY{n}{w}
        \PY{n+nb+bp}{self}\PY{o}{.}\PY{n}{a} \PY{o}{=} \PY{n}{a}
        \PY{n+nb+bp}{self}\PY{o}{.}\PY{n}{b} \PY{o}{=} \PY{n}{b}
        \PY{n+nb+bp}{self}\PY{o}{.}\PY{n}{x0} \PY{o}{=} \PY{n}{x0}
        \PY{n+nb+bp}{self}\PY{o}{.}\PY{n}{delta} \PY{o}{=} \PY{n}{delta}
        \PY{n+nb+bp}{self}\PY{o}{.}\PY{n}{y} \PY{o}{=} \PY{n}{y}
        \PY{n+nb+bp}{self}\PY{o}{.}\PY{n}{dic} \PY{o}{=} \PY{n+nb}{dict}\PY{p}{(}\PY{n+nb}{zip}\PY{p}{(}\PY{n}{t}\PY{p}{,}\PY{n}{y}\PY{p}{)}\PY{p}{)}
        \PY{n+nb+bp}{self}\PY{o}{.}\PY{n}{show\PYZus{}x} \PY{o}{=}  \PY{n}{t}
        \PY{n+nb+bp}{self}\PY{o}{.}\PY{n}{show\PYZus{}y} \PY{o}{=} \PY{n}{y}
\end{Verbatim}
\end{frame}


\begin{frame}{PC-Exercise-9.2.1}
区間$[0,1]$に於いて等間隔$\Delta = 2^{-2}$のオイラー近似を作成し,$dX_t = 1.5 X_t dt + 1.0 X_t dW_t ,\ X_0 =1.0$という確率微分方程式を近似せよ.
\end{frame}

\begin{frame}[containsverbatim]
\begin{Verbatim}[commandchars=\\\{\}]
\PY{n}{a921} \PY{o}{=} \PY{k}{lambda} \PY{n}{t}\PY{p}{,}\PY{n}{x}\PY{p}{:} \PY{l+m+mf}{1.5}\PY{o}{*}\PY{n}{x}
\PY{n}{b921} \PY{o}{=} \PY{k}{lambda} \PY{n}{t}\PY{p}{,}\PY{n}{x}\PY{p}{:} \PY{l+m+mf}{1.0}\PY{o}{*}\PY{n}{x}
\PY{n}{delta921} \PY{o}{=} \PY{l+m+mi}{2}\PY{o}{*}\PY{o}{*}\PY{p}{(}\PY{o}{\PYZhy{}}\PY{l+m+mi}{2}\PY{p}{)}
\PY{n}{x0\PYZus{}921} \PY{o}{=} \PY{l+m+mf}{1.0}
\PY{n}{t0\PYZus{}921} \PY{o}{=} \PY{l+m+mi}{0}
\PY{n}{t1\PYZus{}921} \PY{o}{=} \PY{l+m+mi}{1}

\PY{n}{sol921} \PY{o}{=} \PY{k}{lambda} \PY{n}{t}\PY{p}{,}\PY{n}{w}\PY{p}{:} \PY{n}{math}\PY{o}{.}\PY{n}{exp}\PY{p}{(}\PY{n}{t}\PY{o}{+}\PY{n}{w}\PY{o}{.}\PY{n}{at}\PY{p}{(}\PY{n}{t}\PY{p}{)}\PY{p}{)}

\PY{n}{W} \PY{o}{=} \PY{n}{Wiener}\PY{p}{(}\PY{n}{t0\PYZus{}921}\PY{p}{,}\PY{n}{t1\PYZus{}921}\PY{p}{,}\PY{l+m+mi}{2}\PY{o}{*}\PY{o}{*}\PY{l+m+mi}{9}\PY{p}{)}
\PY{n}{Y} \PY{o}{=} \PY{n}{Euler\PYZus{}Maruyama}\PY{p}{(}\PY{n}{a921}\PY{p}{,}\PY{n}{b921}\PY{p}{,}\PY{n}{x0\PYZus{}921}\PY{p}{,}\PY{n}{W}\PY{p}{,}\PY{n}{delta921}\PY{p}{)}
\PY{n}{X} \PY{o}{=} \PY{n}{Solution}\PY{p}{(}\PY{n}{W}\PY{o}{.}\PY{n}{t}\PY{p}{,}\PY{n}{sol921}\PY{p}{,}\PY{n}{W}\PY{p}{)}

\PY{n}{plt}\PY{o}{.}\PY{n}{title}\PY{p}{(}\PY{l+s+s2}{\PYZdq{}}\PY{l+s+s2}{Euler approximation and exact solution from PC\PYZhy{}Exercise 9.2.1}\PY{l+s+s2}{\PYZdq{}}\PY{p}{)}
\PY{n}{Y}\PY{o}{.}\PY{n}{show}\PY{p}{(}\PY{p}{)}
\PY{n}{X}\PY{o}{.}\PY{n}{show}\PY{p}{(}\PY{p}{)}
\PY{n}{plt}\PY{o}{.}\PY{n}{show}\PY{p}{(}\PY{p}{)}
\PY{n}{plt}\PY{o}{.}\PY{n}{close}\PY{p}{(}\PY{p}{)}
\end{Verbatim}
\end{frame}


\begin{frame}        
\begin{addmargin}[\cellleftmargin]{0em}% left, right
    {\smaller%
    \vspace{-1\smallerfontscale}%
    
    \begin{center}
    \adjustimage{max size={0.9\linewidth}{0.9\paperheight}}{PC-ExerciseChapter9_files/PC-ExerciseChapter9_2_0.png}
    \end{center}
    { \hspace*{\fill} \\}
    }%
\end{addmargin}%
\end{frame}


\begin{frame}{PC-Exercise 9.2.2}
PC-Exercise 9.2.1を間隔を\(\Delta = 2^{-4}\)にして繰り返せ.
\end{frame}

\begin{frame}[containsverbatim]

\begin{Verbatim}[commandchars=\\\{\}]

\PY{n}{delta922} \PY{o}{=} \PY{l+m+mi}{2}\PY{o}{*}\PY{o}{*}\PY{p}{(}\PY{o}{\PYZhy{}}\PY{l+m+mi}{4}\PY{p}{)}
\PY{n}{W} \PY{o}{=} \PY{n}{Wiener}\PY{p}{(}\PY{n}{t0\PYZus{}921}\PY{p}{,}\PY{n}{t1\PYZus{}921}\PY{p}{,}\PY{l+m+mi}{2}\PY{o}{*}\PY{o}{*}\PY{l+m+mi}{9}\PY{p}{)}
\PY{n}{Y} \PY{o}{=} \PY{n}{Euler\PYZus{}Maruyama}\PY{p}{(}\PY{n}{a921}\PY{p}{,}\PY{n}{b921}\PY{p}{,}\PY{n}{x0\PYZus{}921}\PY{p}{,}\PY{n}{W}\PY{p}{,}\PY{n}{delta922}\PY{p}{)}
\PY{n}{X} \PY{o}{=} \PY{n}{Solution}\PY{p}{(}\PY{n}{W}\PY{o}{.}\PY{n}{t}\PY{p}{,}\PY{n}{sol921}\PY{p}{,}\PY{n}{W}\PY{p}{)}

\PY{n}{plt}\PY{o}{.}\PY{n}{title}\PY{p}{(}\PY{l+s+s2}{\PYZdq{}}\PY{l+s+s2}{The Euler approximation for the step size Δ=2\PYZca{}\PYZhy{}4}\PY{l+s+s2}{\PYZdq{}}\PY{p}{)}

\PY{n}{Y}\PY{o}{.}\PY{n}{show}\PY{p}{(}\PY{p}{)}
\PY{n}{X}\PY{o}{.}\PY{n}{show}\PY{p}{(}\PY{p}{)}
\PY{n}{plt}\PY{o}{.}\PY{n}{show}\PY{p}{(}\PY{p}{)}
\PY{n}{plt}\PY{o}{.}\PY{n}{close}\PY{p}{(}\PY{p}{)}

\end{Verbatim}

\end{frame}
\begin{frame}    
    \begin{addmargin}[\cellleftmargin]{0em}% left, right
    {\smaller%
    \vspace{-1\smallerfontscale}%
    
    \begin{center}
    \adjustimage{max size={0.9\linewidth}{0.9\paperheight}}{PC-ExerciseChapter9_files/PC-ExerciseChapter9_4_0.png}
    \end{center}
    { \hspace*{\fill} \\}
    }%
    \end{addmargin}%
\end{frame}

\begin{frame}{PC-Exercise 9.3.1}

\[
dX_t = 1.5 X_t dt + 1.0 X_t dW_t ,\ X_0 =1.0
\]
という$[0,1]$上の伊藤過程に対して$\Delta= 2^{-4}$の等間隔のオイラー近似を$N=25$回繰り返し,統計的誤差\(\hat{\epsilon}\)を計算せよ.これを$\Delta = 2^{-5} , 2^{-6},2^{-7}$ についても繰り返し$\Delta$と$\hat{\epsilon}$の関係を示せ.


\end{frame}
\begin{frame}[containsverbatim]

\begin{Verbatim}[commandchars=\\\{\}]
\PY{n}{N931} \PY{o}{=} \PY{l+m+mi}{100}
\PY{n}{Deltas} \PY{o}{=} \PY{p}{[}\PY{l+m+mi}{2}\PY{o}{*}\PY{o}{*}\PY{p}{(}\PY{o}{\PYZhy{}}\PY{l+m+mi}{4}\PY{p}{)}\PY{p}{,}\PY{l+m+mi}{2}\PY{o}{*}\PY{o}{*}\PY{p}{(}\PY{o}{\PYZhy{}}\PY{l+m+mi}{5}\PY{p}{)}\PY{p}{,}\PY{l+m+mi}{2}\PY{o}{*}\PY{o}{*}\PY{p}{(}\PY{o}{\PYZhy{}}\PY{l+m+mi}{6}\PY{p}{)}\PY{p}{,}\PY{l+m+mi}{2}\PY{o}{*}\PY{o}{*}\PY{p}{(}\PY{o}{\PYZhy{}}\PY{l+m+mi}{7}\PY{p}{)}\PY{p}{]}
\PY{n}{LogDeltas} \PY{o}{=} \PY{p}{[}\PY{l+m+mi}{4}\PY{p}{,}\PY{l+m+mi}{5}\PY{p}{,}\PY{l+m+mi}{6}\PY{p}{,}\PY{l+m+mi}{7}\PY{p}{]}
\PY{n}{epsilons} \PY{o}{=} \PY{p}{[}\PY{p}{]}
\PY{k}{for} \PY{n}{delta} \PY{o+ow}{in} \PY{n}{Deltas}\PY{p}{:}
    \PY{n}{eps} \PY{o}{=} \PY{p}{[}\PY{p}{]}
    
    \PY{k}{for} \PY{n}{i} \PY{o+ow}{in} \PY{n+nb}{range}\PY{p}{(}\PY{n}{N931}\PY{p}{)}\PY{p}{:}
        \PY{n}{W} \PY{o}{=} \PY{n}{Wiener}\PY{p}{(}\PY{n}{t0\PYZus{}921}\PY{p}{,}\PY{n}{t1\PYZus{}921}\PY{p}{,}\PY{l+m+mi}{2}\PY{o}{*}\PY{o}{*}\PY{l+m+mi}{9}\PY{p}{)}
        \PY{n}{Y} \PY{o}{=} \PY{n}{Euler\PYZus{}Maruyama}\PY{p}{(}\PY{n}{a921}\PY{p}{,}\PY{n}{b921}\PY{p}{,}\PY{n}{x0\PYZus{}921}\PY{p}{,}\PY{n}{W}\PY{p}{,}\PY{n}{delta}\PY{p}{)}
        \PY{n}{X} \PY{o}{=} \PY{n}{Solution}\PY{p}{(}\PY{n}{W}\PY{o}{.}\PY{n}{t}\PY{p}{,}\PY{n}{sol921}\PY{p}{,}\PY{n}{W}\PY{p}{)}
        \PY{n}{eps}\PY{o}{.}\PY{n}{append}\PY{p}{(}\PY{n}{math}\PY{o}{.}\PY{n}{fabs}\PY{p}{(}\PY{n}{X}\PY{o}{.}\PY{n}{at}\PY{p}{(}\PY{l+m+mi}{1}\PY{p}{)} \PY{o}{\PYZhy{}} \PY{n}{Y}\PY{o}{.}\PY{n}{at}\PY{p}{(}\PY{l+m+mi}{1}\PY{p}{)}\PY{p}{)}\PY{p}{)}
        
    \PY{n}{epsilons}\PY{o}{.}\PY{n}{append}\PY{p}{(}\PY{n}{np}\PY{o}{.}\PY{n}{mean}\PY{p}{(}\PY{n}{eps}\PY{p}{)}\PY{p}{)}

\PY{n}{plt}\PY{o}{.}\PY{n}{title}\PY{p}{(}\PY{l+s+s2}{\PYZdq{}}\PY{l+s+s2}{Absolute errors for different step lengths}\PY{l+s+s2}{\PYZdq{}}\PY{p}{)}
\PY{n}{plt}\PY{o}{.}\PY{n}{plot}\PY{p}{(}\PY{n}{LogDeltas}\PY{p}{,}\PY{n}{epsilons}\PY{p}{,}\PY{l+s+s2}{\PYZdq{}}\PY{l+s+s2}{\PYZhy{}o}\PY{l+s+s2}{\PYZdq{}}\PY{p}{)}
\PY{n}{plt}\PY{o}{.}\PY{n}{show}\PY{p}{(}\PY{p}{)}
\PY{n}{plt}\PY{o}{.}\PY{n}{close}\PY{p}{(}\PY{p}{)}
\end{Verbatim}


\end{frame}
\begin{frame}
    \begin{addmargin}[\cellleftmargin]{0em}% left, right
    {\smaller%
    \vspace{-1\smallerfontscale}%
    
    \begin{center}
    \adjustimage{max size={0.9\linewidth}{0.9\paperheight}}{PC-ExerciseChapter9_files/PC-ExerciseChapter9_6_0.png}
    \end{center}
    { \hspace*{\fill} \\}
    }%
    \end{addmargin}%
\end{frame}

\begin{frame}{PC-Exercise 9.3.3}
\[
dX_t = 1.5 X_t dt + 0.1 X_t dW_t ,\ X_0 =1.0
\]
という$[0,1]$上の伊藤過程に対して$\Delta= 2^{-4}$の等間隔のオイラー近似を$N=25$回繰り返し,更にこれを$M=10$組繰り返すことによって,絶対誤差 $\epsilon$の90%信頼区間を図示せよ.
これを$M = 20, 40$ and $100$ についても繰り返し$\Delta$と $\epsilon$の信頼区間との関係を示せ.
\end{frame}



\begin{frame}[containsverbatim]

\begin{Verbatim}[commandchars=\\\{\}]
\PY{n}{Ms} \PY{o}{=} \PY{p}{[}\PY{l+m+mi}{10}\PY{p}{,}\PY{l+m+mi}{20}\PY{p}{,}\PY{l+m+mi}{40}\PY{p}{,}\PY{l+m+mi}{100}\PY{p}{]}
\PY{n}{N}  \PY{o}{=} \PY{l+m+mi}{100}
\PY{n}{b933} \PY{o}{=} \PY{k}{lambda} \PY{n}{t}\PY{p}{,}\PY{n}{x}\PY{p}{:} \PY{l+m+mf}{0.1}\PY{o}{*}\PY{n}{x}
\PY{n}{sol933} \PY{o}{=} \PY{k}{lambda} \PY{n}{t}\PY{p}{,}\PY{n}{w}\PY{p}{:} \PY{n}{math}\PY{o}{.}\PY{n}{exp}\PY{p}{(}\PY{l+m+mf}{1.495}\PY{o}{*}\PY{n}{t}\PY{o}{+}\PY{l+m+mf}{0.1}\PY{o}{*}\PY{n}{w}\PY{o}{.}\PY{n}{at}\PY{p}{(}\PY{n}{t}\PY{p}{)}\PY{p}{)}

\PY{n}{all\PYZus{}eps} \PY{o}{=}\PY{p}{[}\PY{p}{]}
\PY{n}{all\PYZus{}deltaeps} \PY{o}{=}\PY{p}{[}\PY{p}{]}
\PY{k}{for} \PY{n}{M} \PY{o+ow}{in}  \PY{n}{Ms}\PY{p}{:}
    \PY{n+nb}{print}\PY{p}{(}\PY{n}{M}\PY{p}{)}
    \PY{n}{epsilons} \PY{o}{=}\PY{p}{[}\PY{p}{]}
    \PY{k}{for} \PY{n}{i} \PY{o+ow}{in} \PY{n+nb}{range}\PY{p}{(}\PY{n}{M}\PY{p}{)}\PY{p}{:}
        \PY{n}{eps} \PY{o}{=} \PY{p}{[}\PY{p}{]}
        \PY{k}{for} \PY{n}{j} \PY{o+ow}{in} \PY{n+nb}{range}\PY{p}{(}\PY{n}{N}\PY{p}{)}\PY{p}{:}
            \PY{n}{W} \PY{o}{=} \PY{n}{Wiener}\PY{p}{(}\PY{n}{t0\PYZus{}921}\PY{p}{,}\PY{n}{t1\PYZus{}921}\PY{p}{,}\PY{l+m+mi}{2}\PY{o}{*}\PY{o}{*}\PY{l+m+mi}{9}\PY{p}{)}
            \PY{n}{Y} \PY{o}{=} \PY{n}{Euler\PYZus{}Maruyama}\PY{p}{(}\PY{n}{a921}\PY{p}{,}\PY{n}{b933}\PY{p}{,}\PY{n}{x0\PYZus{}921}\PY{p}{,}\PY{n}{W}\PY{p}{,}\PY{n}{delta}\PY{p}{)}
            \PY{n}{eps}\PY{o}{.}\PY{n}{append}\PY{p}{(}\PY{n}{math}\PY{o}{.}\PY{n}{fabs}\PY{p}{(}\PY{n}{sol933}\PY{p}{(}\PY{l+m+mi}{1}\PY{p}{,}\PY{n}{W}\PY{p}{)} \PY{o}{\PYZhy{}} \PY{n}{Y}\PY{o}{.}\PY{n}{at}\PY{p}{(}\PY{l+m+mi}{1}\PY{p}{)}\PY{p}{)}\PY{p}{)}
        \PY{n}{epsilons}\PY{o}{.}\PY{n}{append}\PY{p}{(}\PY{n}{np}\PY{o}{.}\PY{n}{mean}\PY{p}{(}\PY{n}{eps}\PY{p}{)}\PY{p}{)}
    \PY{n}{all\PYZus{}eps}\PY{o}{.}\PY{n}{append}\PY{p}{(}\PY{n}{np}\PY{o}{.}\PY{n}{mean}\PY{p}{(}\PY{n}{epsilons}\PY{p}{)}\PY{p}{)}
    \PY{n}{all\PYZus{}deltaeps}\PY{o}{.}\PY{n}{append}\PY{p}{(}\PY{n}{stats}\PY{o}{.}\PY{n}{t}\PY{o}{.}\PY{n}{ppf}\PY{p}{(}\PY{l+m+mi}{1}\PY{o}{\PYZhy{}}\PY{p}{(}\PY{l+m+mi}{1}\PY{o}{\PYZhy{}}\PY{l+m+mf}{0.9}\PY{p}{)}\PY{o}{/}\PY{l+m+mi}{2}\PY{p}{,} \PY{n}{M}\PY{o}{\PYZhy{}}\PY{l+m+mi}{1}\PY{p}{)} \PY{o}{*} \PY{n}{math}\PY{o}{.}\PY{n}{sqrt}\PY{p}{(}\PY{n}{np}\PY{o}{.}\PY{n}{var}\PY{p}{(}\PY{n}{epsilons}\PY{p}{)}\PY{o}{/}\PY{n}{M}\PY{p}{)}\PY{p}{)}

\PY{n}{plt}\PY{o}{.}\PY{n}{errorbar}\PY{p}{(}\PY{n}{Ms}\PY{p}{,}\PY{n}{all\PYZus{}eps}\PY{p}{,}\PY{n}{yerr}\PY{o}{=}\PY{n}{all\PYZus{}deltaeps}\PY{p}{)}
\PY{n}{plt}\PY{o}{.}\PY{n}{xlim}\PY{p}{(}\PY{l+m+mi}{0}\PY{p}{,}\PY{l+m+mi}{120}\PY{p}{)}
\PY{n}{plt}\PY{o}{.}\PY{n}{show}\PY{p}{(}\PY{p}{)}
\PY{n}{plt}\PY{o}{.}\PY{n}{close}\PY{p}{(}\PY{p}{)}
\end{Verbatim}

 
\end{frame}
\begin{frame}
\begin{addmargin}[\cellleftmargin]{0em}% left, right
    {\smaller%
    \vspace{-1\smallerfontscale}%
    
    \begin{center}
    \adjustimage{max size={0.9\linewidth}{0.9\paperheight}}{PC-ExerciseChapter9_files/PC-ExerciseChapter9_8_1.png}
    \end{center}
    { \hspace*{\fill} \\}
    }%
    \end{addmargin}%
	
\end{frame}

\begin{frame}{PC-Exercise 9.3.4}
\[
dX_t = 1.5 X_t dt + 0.1 X_t dW_t ,\ X_0 =1.0
\]
という$[0,1]$上の伊藤過程に対して$\Delta= 2^{-2}$の等間隔のオイラー近似を$N=25$回繰り返し,更にこれを$M=100$組繰り返すことによって,絶対誤差 $\epsilon$の90%信頼区間を図示せよ.これを,$\Delta . = 2^{-5} , 2^{-6}$ and $2^{-7} $についても繰り返し$\Delta$と $\epsilon$の関係を示せ.
\end{frame}
\begin{frame}[containsverbatim]
\begin{Verbatim}[commandchars=\\\{\}]
\PY{n}{Deltas} \PY{o}{=} \PY{p}{[}\PY{l+m+mi}{2}\PY{o}{*}\PY{o}{*}\PY{p}{(}\PY{o}{\PYZhy{}}\PY{l+m+mi}{1}\PY{p}{)}\PY{p}{,}\PY{l+m+mi}{2}\PY{o}{*}\PY{o}{*}\PY{p}{(}\PY{o}{\PYZhy{}}\PY{l+m+mi}{2}\PY{p}{)}\PY{p}{,}\PY{l+m+mi}{2}\PY{o}{*}\PY{o}{*}\PY{p}{(}\PY{o}{\PYZhy{}}\PY{l+m+mi}{3}\PY{p}{)}\PY{p}{,}\PY{l+m+mi}{2}\PY{o}{*}\PY{o}{*}\PY{p}{(}\PY{o}{\PYZhy{}}\PY{l+m+mi}{4}\PY{p}{)}\PY{p}{,}\PY{l+m+mi}{2}\PY{o}{*}\PY{o}{*}\PY{p}{(}\PY{o}{\PYZhy{}}\PY{l+m+mi}{5}\PY{p}{)}\PY{p}{]}
\PY{n}{LogDeltas} \PY{o}{=} \PY{p}{[}\PY{o}{\PYZhy{}}\PY{l+m+mi}{1}\PY{p}{,}\PY{o}{\PYZhy{}}\PY{l+m+mi}{2}\PY{p}{,}\PY{o}{\PYZhy{}}\PY{l+m+mi}{3}\PY{p}{,}\PY{o}{\PYZhy{}}\PY{l+m+mi}{4}\PY{p}{,}\PY{o}{\PYZhy{}}\PY{l+m+mi}{5}\PY{p}{]}
\PY{n}{N}  \PY{o}{=} \PY{l+m+mi}{100}
\PY{n}{M} \PY{o}{=} \PY{l+m+mi}{20}

\PY{n}{all\PYZus{}eps} \PY{o}{=}\PY{p}{[}\PY{p}{]}
\PY{n}{all\PYZus{}deltaeps} \PY{o}{=}\PY{p}{[}\PY{p}{]}
\PY{k}{for} \PY{n}{delta} \PY{o+ow}{in}  \PY{n}{Deltas}\PY{p}{:}
    \PY{n+nb}{print}\PY{p}{(}\PY{n}{delta}\PY{p}{)}
    \PY{n}{epsilons} \PY{o}{=}\PY{p}{[}\PY{p}{]}
    \PY{k}{for} \PY{n}{i} \PY{o+ow}{in} \PY{n+nb}{range}\PY{p}{(}\PY{n}{M}\PY{p}{)}\PY{p}{:}
        \PY{n}{eps} \PY{o}{=} \PY{p}{[}\PY{p}{]}
        \PY{k}{for} \PY{n}{j} \PY{o+ow}{in} \PY{n+nb}{range}\PY{p}{(}\PY{n}{N}\PY{p}{)}\PY{p}{:}
            \PY{n}{W} \PY{o}{=} \PY{n}{Wiener}\PY{p}{(}\PY{n}{t0\PYZus{}921}\PY{p}{,}\PY{n}{t1\PYZus{}921}\PY{p}{,}\PY{l+m+mi}{2}\PY{o}{*}\PY{o}{*}\PY{l+m+mi}{10}\PY{p}{)}
            \PY{n}{Y} \PY{o}{=} \PY{n}{Euler\PYZus{}Maruyama}\PY{p}{(}\PY{n}{a921}\PY{p}{,}\PY{n}{b933}\PY{p}{,}\PY{n}{x0\PYZus{}921}\PY{p}{,}\PY{n}{W}\PY{p}{,}\PY{n}{delta}\PY{p}{)}
            \PY{n}{eps}\PY{o}{.}\PY{n}{append}\PY{p}{(}\PY{n}{math}\PY{o}{.}\PY{n}{fabs}\PY{p}{(}\PY{n}{sol933}\PY{p}{(}\PY{l+m+mi}{1}\PY{p}{,}\PY{n}{W}\PY{p}{)} \PY{o}{\PYZhy{}} \PY{n}{Y}\PY{o}{.}\PY{n}{at}\PY{p}{(}\PY{l+m+mi}{1}\PY{p}{)}\PY{p}{)}\PY{p}{)}
        \PY{n}{epsilons}\PY{o}{.}\PY{n}{append}\PY{p}{(}\PY{n}{np}\PY{o}{.}\PY{n}{mean}\PY{p}{(}\PY{n}{eps}\PY{p}{)}\PY{p}{)}
    \PY{n}{all\PYZus{}eps}\PY{o}{.}\PY{n}{append}\PY{p}{(}\PY{n}{np}\PY{o}{.}\PY{n}{mean}\PY{p}{(}\PY{n}{epsilons}\PY{p}{)}\PY{p}{)}
    \PY{n}{all\PYZus{}deltaeps}\PY{o}{.}\PY{n}{append}\PY{p}{(}\PY{n}{stats}\PY{o}{.}\PY{n}{t}\PY{o}{.}\PY{n}{ppf}\PY{p}{(}\PY{l+m+mi}{1}\PY{o}{\PYZhy{}}\PY{p}{(}\PY{l+m+mi}{1}\PY{o}{\PYZhy{}}\PY{l+m+mf}{0.9}\PY{p}{)}\PY{o}{/}\PY{l+m+mi}{2}\PY{p}{,} \PY{n}{M}\PY{o}{\PYZhy{}}\PY{l+m+mi}{1}\PY{p}{)} \PY{o}{*} \PY{n}{math}\PY{o}{.}\PY{n}{sqrt}\PY{p}{(}\PY{n}{np}\PY{o}{.}\PY{n}{var}\PY{p}{(}\PY{n}{epsilons}\PY{p}{)}\PY{o}{/}\PY{n}{M}\PY{p}{)}\PY{p}{)}



\PY{n}{plt}\PY{o}{.}\PY{n}{errorbar}\PY{p}{(}\PY{n}{LogDeltas}\PY{p}{,}\PY{n}{all\PYZus{}eps}\PY{p}{,}\PY{n}{yerr}\PY{o}{=}\PY{n}{all\PYZus{}deltaeps}\PY{p}{)}

\PY{n}{plt}\PY{o}{.}\PY{n}{show}\PY{p}{(}\PY{p}{)}
\PY{n}{plt}\PY{o}{.}\PY{n}{close}\PY{p}{(}\PY{p}{)}
\end{Verbatim}
%
\end{frame}
\begin{frame} 
    \begin{addmargin}[\cellleftmargin]{0em}% left, right
    {\smaller%
    \vspace{-1\smallerfontscale}%
    
    \begin{center}
    \adjustimage{max size={0.9\linewidth}{0.9\paperheight}}{PC-ExerciseChapter9_files/PC-ExerciseChapter9_10_1.png}
    \end{center}
    { \hspace*{\fill} \\}
    }%
    \end{addmargin}%
\end{frame}

\begin{frame}{PC-Exercise 9.4.1}
$dX_t = 1.5 X_t dt + 0.1 X_t dW_t ,\ X_0 =1.0$という$[0,1]$上の伊藤過程に対して$\Delta= 2^{-4}$の等間隔のオイラー近似を$N=100$回繰り返し,
更にこれを$M=20,40,100$組繰り返すことによって,平均の誤差 $\mu$の90%信頼区間を図示せよ.
\end{frame}


\begin{frame}[containsverbatim]
\begin{Verbatim}[commandchars=\\\{\}]
\PY{n}{Ms} \PY{o}{=} \PY{p}{[}\PY{l+m+mi}{10}\PY{p}{,}\PY{l+m+mi}{20}\PY{p}{,}\PY{l+m+mi}{40}\PY{p}{,}\PY{l+m+mi}{100}\PY{p}{]}
\PY{n}{N}  \PY{o}{=} \PY{l+m+mi}{100}
\PY{n}{delta} \PY{o}{=} \PY{l+m+mi}{2}\PY{o}{*}\PY{o}{*}\PY{p}{(}\PY{o}{\PYZhy{}}\PY{l+m+mi}{4}\PY{p}{)}
\PY{n}{mus} \PY{o}{=}\PY{p}{[}\PY{p}{]}
\PY{n}{delta\PYZus{}mu} \PY{o}{=}\PY{p}{[}\PY{p}{]}

\PY{k}{for} \PY{n}{M} \PY{o+ow}{in}  \PY{n}{Ms}\PY{p}{:}
    \PY{n+nb}{print}\PY{p}{(}\PY{n}{M}\PY{p}{)}
    \PY{n}{mu\PYZus{}j} \PY{o}{=}\PY{p}{[}\PY{p}{]}
    \PY{k}{for} \PY{n}{i} \PY{o+ow}{in} \PY{n+nb}{range}\PY{p}{(}\PY{n}{M}\PY{p}{)}\PY{p}{:}
        \PY{n}{Y\PYZus{}t\PYZus{}j} \PY{o}{=} \PY{p}{[}\PY{p}{]}
        \PY{k}{for} \PY{n}{j} \PY{o+ow}{in} \PY{n+nb}{range}\PY{p}{(}\PY{n}{N}\PY{p}{)}\PY{p}{:}
            \PY{n}{W} \PY{o}{=} \PY{n}{Wiener}\PY{p}{(}\PY{n}{t0\PYZus{}921}\PY{p}{,}\PY{n}{t1\PYZus{}921}\PY{p}{,}\PY{l+m+mi}{2}\PY{o}{*}\PY{o}{*}\PY{l+m+mi}{9}\PY{p}{)}
            \PY{n}{Y} \PY{o}{=} \PY{n}{Euler\PYZus{}Maruyama}\PY{p}{(}\PY{n}{a921}\PY{p}{,}\PY{n}{b933}\PY{p}{,}\PY{n}{x0\PYZus{}921}\PY{p}{,}\PY{n}{W}\PY{p}{,}\PY{n}{delta}\PY{p}{)}
            \PY{n}{Y\PYZus{}t\PYZus{}j}\PY{o}{.}\PY{n}{append}\PY{p}{(}\PY{n}{Y}\PY{o}{.}\PY{n}{at}\PY{p}{(}\PY{l+m+mi}{1}\PY{p}{)}\PY{p}{)}
        \PY{n}{mu\PYZus{}j}\PY{o}{.}\PY{n}{append}\PY{p}{(}\PY{n}{np}\PY{o}{.}\PY{n}{mean}\PY{p}{(}\PY{n}{Y\PYZus{}t\PYZus{}j}\PY{p}{)}\PY{o}{\PYZhy{}}\PY{n}{math}\PY{o}{.}\PY{n}{exp}\PY{p}{(}\PY{l+m+mf}{1.5}\PY{p}{)}\PY{p}{)}
    \PY{n}{mus}\PY{o}{.}\PY{n}{append}\PY{p}{(}\PY{n}{np}\PY{o}{.}\PY{n}{mean}\PY{p}{(}\PY{n}{mu\PYZus{}j}\PY{p}{)}\PY{p}{)}
    \PY{n}{delta\PYZus{}mu}\PY{o}{.}\PY{n}{append}\PY{p}{(}\PY{n}{stats}\PY{o}{.}\PY{n}{t}\PY{o}{.}\PY{n}{ppf}\PY{p}{(}\PY{l+m+mi}{1}\PY{o}{\PYZhy{}}\PY{p}{(}\PY{l+m+mi}{1}\PY{o}{\PYZhy{}}\PY{l+m+mf}{0.9}\PY{p}{)}\PY{o}{/}\PY{l+m+mi}{2}\PY{p}{,} \PY{n}{M}\PY{o}{\PYZhy{}}\PY{l+m+mi}{1}\PY{p}{)} \PY{o}{*} \PY{n}{math}\PY{o}{.}\PY{n}{sqrt}\PY{p}{(}\PY{n}{np}\PY{o}{.}\PY{n}{var}\PY{p}{(}\PY{n}{mu\PYZus{}j}\PY{p}{)}\PY{o}{/}\PY{n}{M}\PY{p}{)}\PY{p}{)}
                    
\PY{n}{plt}\PY{o}{.}\PY{n}{errorbar}\PY{p}{(}\PY{n}{Ms}\PY{p}{,}\PY{n}{mus}\PY{p}{,}\PY{n}{yerr}\PY{o}{=}\PY{n}{delta\PYZus{}mu}\PY{p}{)}
\PY{n}{plt}\PY{o}{.}\PY{n}{xlim}\PY{p}{(}\PY{l+m+mi}{0}\PY{p}{,}\PY{l+m+mi}{120}\PY{p}{)}
\PY{n}{plt}\PY{o}{.}\PY{n}{show}\PY{p}{(}\PY{p}{)}
\PY{n}{plt}\PY{o}{.}\PY{n}{close}\PY{p}{(}\PY{p}{)}
\end{Verbatim}
\end{frame}

\begin{frame}
    \begin{addmargin}[\cellleftmargin]{0em}% left, right
    {\smaller%
    \vspace{-1\smallerfontscale}%
    
    \begin{center}
    \adjustimage{max size={0.9\linewidth}{0.9\paperheight}}{PC-ExerciseChapter9_files/PC-ExerciseChapter9_12_1.png}
    \end{center}
    { \hspace*{\fill} \\}
    }%
    \end{addmargin}%

\end{frame}
\begin{frame}
$dX_t = 1.5 X_t dt + 0.1 X_t dW_t ,\ X_0 =1.0$という$[0,1]$上の伊藤過程に対して,
$M=20,N=100$として,$\Delta = 2^{-3},\cdots,2^{-5}$に対して$\mu$の信頼区間を示せ.
\end{frame}

\begin{frame}[containsverbatim]

\begin{Verbatim}[commandchars=\\\{\}]
\PY{n}{Deltas} \PY{o}{=} \PY{p}{[}\PY{l+m+mi}{2}\PY{o}{*}\PY{o}{*}\PY{p}{(}\PY{o}{\PYZhy{}}\PY{l+m+mi}{1}\PY{p}{)}\PY{p}{,}\PY{l+m+mi}{2}\PY{o}{*}\PY{o}{*}\PY{p}{(}\PY{o}{\PYZhy{}}\PY{l+m+mi}{2}\PY{p}{)}\PY{p}{,}\PY{l+m+mi}{2}\PY{o}{*}\PY{o}{*}\PY{p}{(}\PY{o}{\PYZhy{}}\PY{l+m+mi}{3}\PY{p}{)}\PY{p}{,}\PY{l+m+mi}{2}\PY{o}{*}\PY{o}{*}\PY{p}{(}\PY{o}{\PYZhy{}}\PY{l+m+mi}{4}\PY{p}{)}\PY{p}{,}\PY{l+m+mi}{2}\PY{o}{*}\PY{o}{*}\PY{p}{(}\PY{o}{\PYZhy{}}\PY{l+m+mi}{5}\PY{p}{)}\PY{p}{]}
\PY{n}{LogDeltas} \PY{o}{=} \PY{p}{[}\PY{o}{\PYZhy{}}\PY{l+m+mi}{1}\PY{p}{,}\PY{o}{\PYZhy{}}\PY{l+m+mi}{2}\PY{p}{,}\PY{o}{\PYZhy{}}\PY{l+m+mi}{3}\PY{p}{,}\PY{o}{\PYZhy{}}\PY{l+m+mi}{4}\PY{p}{,}\PY{o}{\PYZhy{}}\PY{l+m+mi}{5}\PY{p}{]}
\PY{n}{N}  \PY{o}{=} \PY{l+m+mi}{100}
\PY{n}{M} \PY{o}{=} \PY{l+m+mi}{20}

\PY{n}{mus} \PY{o}{=}\PY{p}{[}\PY{p}{]}
\PY{n}{delta\PYZus{}mu} \PY{o}{=}\PY{p}{[}\PY{p}{]}

\PY{k}{for} \PY{n}{delta} \PY{o+ow}{in}  \PY{n}{Deltas}\PY{p}{:}
    \PY{n+nb}{print}\PY{p}{(}\PY{n}{M}\PY{p}{)}
    \PY{n}{mu\PYZus{}j} \PY{o}{=}\PY{p}{[}\PY{p}{]}
    \PY{k}{for} \PY{n}{i} \PY{o+ow}{in} \PY{n+nb}{range}\PY{p}{(}\PY{n}{M}\PY{p}{)}\PY{p}{:}
        \PY{n}{Y\PYZus{}t\PYZus{}j} \PY{o}{=} \PY{p}{[}\PY{p}{]}
        \PY{k}{for} \PY{n}{j} \PY{o+ow}{in} \PY{n+nb}{range}\PY{p}{(}\PY{n}{N}\PY{p}{)}\PY{p}{:}
            \PY{n}{W} \PY{o}{=} \PY{n}{Wiener}\PY{p}{(}\PY{n}{t0\PYZus{}921}\PY{p}{,}\PY{n}{t1\PYZus{}921}\PY{p}{,}\PY{l+m+mi}{2}\PY{o}{*}\PY{o}{*}\PY{l+m+mi}{9}\PY{p}{)}
            \PY{n}{Y} \PY{o}{=} \PY{n}{Euler\PYZus{}Maruyama}\PY{p}{(}\PY{n}{a921}\PY{p}{,}\PY{n}{b933}\PY{p}{,}\PY{n}{x0\PYZus{}921}\PY{p}{,}\PY{n}{W}\PY{p}{,}\PY{n}{delta}\PY{p}{)}
            \PY{n}{Y\PYZus{}t\PYZus{}j}\PY{o}{.}\PY{n}{append}\PY{p}{(}\PY{n}{Y}\PY{o}{.}\PY{n}{at}\PY{p}{(}\PY{l+m+mi}{1}\PY{p}{)}\PY{p}{)}
        \PY{n}{mu\PYZus{}j}\PY{o}{.}\PY{n}{append}\PY{p}{(}\PY{n}{np}\PY{o}{.}\PY{n}{mean}\PY{p}{(}\PY{n}{Y\PYZus{}t\PYZus{}j}\PY{p}{)}\PY{o}{\PYZhy{}}\PY{n}{math}\PY{o}{.}\PY{n}{exp}\PY{p}{(}\PY{l+m+mf}{1.5}\PY{p}{)}\PY{p}{)}
    
    \PY{n}{mus}\PY{o}{.}\PY{n}{append}\PY{p}{(}\PY{n}{np}\PY{o}{.}\PY{n}{mean}\PY{p}{(}\PY{n}{mu\PYZus{}j}\PY{p}{)}\PY{p}{)}
    \PY{n}{delta\PYZus{}mu}\PY{o}{.}\PY{n}{append}\PY{p}{(}\PY{n}{stats}\PY{o}{.}\PY{n}{t}\PY{o}{.}\PY{n}{ppf}\PY{p}{(}\PY{l+m+mi}{1}\PY{o}{\PYZhy{}}\PY{p}{(}\PY{l+m+mi}{1}\PY{o}{\PYZhy{}}\PY{l+m+mf}{0.9}\PY{p}{)}\PY{o}{/}\PY{l+m+mi}{2}\PY{p}{,} \PY{n}{M}\PY{o}{\PYZhy{}}\PY{l+m+mi}{1}\PY{p}{)} \PY{o}{*} \PY{n}{math}\PY{o}{.}\PY{n}{sqrt}\PY{p}{(}\PY{n}{np}\PY{o}{.}\PY{n}{var}\PY{p}{(}\PY{n}{mu\PYZus{}j}\PY{p}{)}\PY{o}{/}\PY{n}{M}\PY{p}{)}\PY{p}{)}
                    
\PY{n}{plt}\PY{o}{.}\PY{n}{errorbar}\PY{p}{(}\PY{n}{Deltas}\PY{p}{,}\PY{n}{mus}\PY{p}{,}\PY{n}{yerr}\PY{o}{=}\PY{n}{delta\PYZus{}mu}\PY{p}{)}
\PY{n}{plt}\PY{o}{.}\PY{n}{show}\PY{p}{(}\PY{p}{)}
\PY{n}{plt}\PY{o}{.}\PY{n}{close}\PY{p}{(}\PY{p}{)}
\end{Verbatim}
\end{frame}

\begin{frame}
    %
    \begin{addmargin}[\cellleftmargin]{0em}% left, right
    {\smaller%
    \vspace{-1\smallerfontscale}%
    
    \begin{center}
    \adjustimage{max size={0.9\linewidth}{0.9\paperheight}}{PC-ExerciseChapter9_files/PC-ExerciseChapter9_13_1.png}
    \end{center}
    { \hspace*{\fill} \\}
    }%
    \end{addmargin}%
\end{frame}


\begin{frame}{PC-Exercise9.8.2}
\[
dX_t = 5 X_t dt + dW_t ,\ X_0 =1.0
\]
という$[0,1]$上の伊藤過程に対して$\Delta= 2^{-4}$の等間隔のオイラー近似をせよ.

\end{frame}
\begin{frame}[containsverbatim]
\begin{Verbatim}[commandchars=\\\{\}]
\PY{n}{a982} \PY{o}{=} \PY{k}{lambda} \PY{n}{t}\PY{p}{,}\PY{n}{x}\PY{p}{:} \PY{l+m+mi}{5}\PY{o}{*}\PY{n}{x}
\PY{n}{b982} \PY{o}{=} \PY{k}{lambda} \PY{n}{t}\PY{p}{,}\PY{n}{x}\PY{p}{:} \PY{l+m+mi}{1}
\PY{n}{delta982} \PY{o}{=} \PY{l+m+mi}{2}\PY{o}{*}\PY{o}{*}\PY{p}{(}\PY{o}{\PYZhy{}}\PY{l+m+mi}{4}\PY{p}{)}
\PY{n}{x0\PYZus{}921} \PY{o}{=} \PY{l+m+mf}{1.0}
\PY{n}{t0\PYZus{}921} \PY{o}{=} \PY{l+m+mi}{0}
\PY{n}{t1\PYZus{}921} \PY{o}{=} \PY{l+m+mi}{1}

\PY{n}{sol921} \PY{o}{=} \PY{k}{lambda} \PY{n}{t}\PY{p}{,}\PY{n}{w}\PY{p}{:} \PY{n}{math}\PY{o}{.}\PY{n}{exp}\PY{p}{(}\PY{n}{t}\PY{o}{+}\PY{n}{w}\PY{o}{.}\PY{n}{at}\PY{p}{(}\PY{n}{t}\PY{p}{)}\PY{p}{)}

\PY{n}{W} \PY{o}{=} \PY{n}{Wiener}\PY{p}{(}\PY{n}{t0\PYZus{}921}\PY{p}{,}\PY{n}{t1\PYZus{}921}\PY{p}{,}\PY{l+m+mi}{2}\PY{o}{*}\PY{o}{*}\PY{l+m+mi}{10}\PY{p}{)}
\PY{n}{Y} \PY{o}{=} \PY{n}{Euler\PYZus{}Maruyama}\PY{p}{(}\PY{n}{a982}\PY{p}{,}\PY{n}{b982}\PY{p}{,}\PY{n}{x0\PYZus{}921}\PY{p}{,}\PY{n}{W}\PY{p}{,}\PY{n}{delta921}\PY{p}{)}
\PY{n}{X} \PY{o}{=} \PY{n}{Euler\PYZus{}Maruyama}\PY{p}{(}\PY{n}{a982}\PY{p}{,}\PY{n}{b982}\PY{p}{,}\PY{n}{x0\PYZus{}921}\PY{p}{,}\PY{n}{W}\PY{p}{,}\PY{l+m+mi}{2}\PY{o}{*}\PY{o}{*}\PY{p}{(}\PY{o}{\PYZhy{}}\PY{l+m+mi}{10}\PY{p}{)}\PY{p}{)}

\PY{n}{plt}\PY{o}{.}\PY{n}{title}\PY{p}{(}\PY{l+s+s2}{\PYZdq{}}\PY{l+s+s2}{Euler approximation and exact solution from PC\PYZhy{}Exercise 9.8.2}\PY{l+s+s2}{\PYZdq{}}\PY{p}{)}
\PY{n}{Y}\PY{o}{.}\PY{n}{show}\PY{p}{(}\PY{p}{)}
\PY{n}{X}\PY{o}{.}\PY{n}{show}\PY{p}{(}\PY{p}{)}
\PY{n}{plt}\PY{o}{.}\PY{n}{show}\PY{p}{(}\PY{p}{)}
\PY{n}{plt}\PY{o}{.}\PY{n}{close}\PY{p}{(}\PY{p}{)}
\end{Verbatim}
\end{frame}

\begin{frame}    \begin{addmargin}[\cellleftmargin]{0em}% left, right
    {\smaller%
    \vspace{-1\smallerfontscale}%
    
    \begin{center}
    \adjustimage{max size={0.9\linewidth}{0.9\paperheight}}{PC-ExerciseChapter9_files/PC-ExerciseChapter9_15_0.png}
    \end{center}
    { \hspace*{\fill} \\}
    }%
    \end{addmargin}%
\end{frame}
\begin{frame}{PC-Exercise9.8.3}
\[
dX_t = -5 X_t dt + dW_t ,\ X_0 =1.0
\]
という$[0,1]$上の伊藤過程に対して$\Delta= 2^{-4}$の等間隔のオイラー近似をせよ.


\end{frame}

\begin{frame}[containsverbatim]
\begin{Verbatim}[commandchars=\\\{\}]
\PY{n}{a982} \PY{o}{=} \PY{k}{lambda} \PY{n}{t}\PY{p}{,}\PY{n}{x}\PY{p}{:} \PY{o}{\PYZhy{}}\PY{l+m+mi}{5}\PY{o}{*}\PY{n}{x}
\PY{n}{b982} \PY{o}{=} \PY{k}{lambda} \PY{n}{t}\PY{p}{,}\PY{n}{x}\PY{p}{:} \PY{l+m+mi}{1}
\PY{n}{delta982} \PY{o}{=} \PY{l+m+mi}{2}\PY{o}{*}\PY{o}{*}\PY{p}{(}\PY{o}{\PYZhy{}}\PY{l+m+mi}{4}\PY{p}{)}
\PY{n}{x0\PYZus{}921} \PY{o}{=} \PY{l+m+mf}{1.0}
\PY{n}{t0\PYZus{}921} \PY{o}{=} \PY{l+m+mi}{0}
\PY{n}{t1\PYZus{}921} \PY{o}{=} \PY{l+m+mi}{1}

\PY{n}{sol921} \PY{o}{=} \PY{k}{lambda} \PY{n}{t}\PY{p}{,}\PY{n}{w}\PY{p}{:} \PY{n}{math}\PY{o}{.}\PY{n}{exp}\PY{p}{(}\PY{n}{t}\PY{o}{+}\PY{n}{w}\PY{o}{.}\PY{n}{at}\PY{p}{(}\PY{n}{t}\PY{p}{)}\PY{p}{)}

\PY{n}{W} \PY{o}{=} \PY{n}{Wiener}\PY{p}{(}\PY{n}{t0\PYZus{}921}\PY{p}{,}\PY{n}{t1\PYZus{}921}\PY{p}{,}\PY{l+m+mi}{2}\PY{o}{*}\PY{o}{*}\PY{l+m+mi}{10}\PY{p}{)}
\PY{n}{Y} \PY{o}{=} \PY{n}{Euler\PYZus{}Maruyama}\PY{p}{(}\PY{n}{a982}\PY{p}{,}\PY{n}{b982}\PY{p}{,}\PY{n}{x0\PYZus{}921}\PY{p}{,}\PY{n}{W}\PY{p}{,}\PY{n}{delta921}\PY{p}{)}
\PY{n}{X} \PY{o}{=} \PY{n}{Euler\PYZus{}Maruyama}\PY{p}{(}\PY{n}{a982}\PY{p}{,}\PY{n}{b982}\PY{p}{,}\PY{n}{x0\PYZus{}921}\PY{p}{,}\PY{n}{W}\PY{p}{,}\PY{l+m+mi}{2}\PY{o}{*}\PY{o}{*}\PY{p}{(}\PY{o}{\PYZhy{}}\PY{l+m+mi}{10}\PY{p}{)}\PY{p}{)}

\PY{n}{plt}\PY{o}{.}\PY{n}{title}\PY{p}{(}\PY{l+s+s2}{\PYZdq{}}\PY{l+s+s2}{Euler approximation and exact solution from PC\PYZhy{}Exercise 9.8.2}\PY{l+s+s2}{\PYZdq{}}\PY{p}{)}
\PY{n}{Y}\PY{o}{.}\PY{n}{show}\PY{p}{(}\PY{p}{)}
\PY{n}{X}\PY{o}{.}\PY{n}{show}\PY{p}{(}\PY{p}{)}
\PY{n}{plt}\PY{o}{.}\PY{n}{show}\PY{p}{(}\PY{p}{)}
\PY{n}{plt}\PY{o}{.}\PY{n}{close}\PY{p}{(}\PY{p}{)}
\end{Verbatim}
\end{frame}


\begin{frame}
    \begin{addmargin}[\cellleftmargin]{0em}% left, right
    {\smaller%
    \vspace{-1\smallerfontscale}%
    
    \begin{center}
    \adjustimage{max size={0.9\linewidth}{0.9\paperheight}}{PC-ExerciseChapter9_files/PC-ExerciseChapter9_17_0.png}
    \end{center}
    { \hspace*{\fill} \\}
    }%
    \end{addmargin}%


\end{frame}
\end{document}