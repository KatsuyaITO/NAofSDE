\documentclass[a4paper,dvipdfmx]{jreport}
\usepackage{amsmath,amssymb}
\usepackage[dvipdfmx]{graphicx}
\usepackage[dvipdfm]{hyperref}
\usepackage{pxjahyper}
\usepackage{framed}
\usepackage{color}

\def\qedsymbol{$\square$}
\def\proofname{\gt{証明}\;}
\newenvironment{Proof}{\par\noindent{\it\proofname}}{{\unskip\nobreak\hfill{\it\qedsymbol}}\par\vskip 9pt}
\newenvironment{Proof*}{\par\noindent}{{\unskip\nobreak\hfill{\it\qedsymbol}}\par\vskip 9pt}
\ifx\undefined\bysame \newcommand{\bysame}{\leavevmode\hbox to3em{\hrulefill}\,}\fi
%
\numberwithin{equation}{section}
\newtheorem{Thm}     {定理}[section]
\newtheorem{Lemma}   [Thm]{補題}
\newtheorem{Def}     [Thm]{定義}
\newtheorem{Prop}    [Thm]{命題}
\newtheorem{Fact}    [Thm]{事実}
\newtheorem{Cor}     [Thm]{系}  
\newtheorem{Conj}    [Thm]{予想}
\newtheorem{Ex}      [Thm]{例}  
\newtheorem{Achiom}   [Thm]{公理}
\newtheorem{Method}[Thm]{方法} 
\newtheorem{Rem}  [Thm]{注意}
\newtheorem{Notation}[Thm]{記法}
\newtheorem{Symbol}  [Thm]{記号}
\newtheorem{Prob}    [Thm]{問題}
\makeatletter
\renewenvironment{leftbar}{%
  \def\FrameCommand{\vrule width 1pt \hspace{10pt}}% 
  \MakeFramed {\advance\hsize-\width \FrameRestore}}%
 {\endMakeFramed}
\makeatother

\newenvironment{redleftbar}{%
  \def\FrameCommand{\textcolor{red}{\vrule width 1pt} \hspace{10pt}}% 
  \MakeFramed {\advance\hsize-\width \FrameRestore}}%
 {\endMakeFramed}

\newenvironment{lightgrayleftbar}{%
  \def\FrameCommand{\textcolor{lightgray}{\vrule width 1zw} \hspace{10pt}}% 
  \MakeFramed {\advance\hsize-\width \FrameRestore}}%
{\endMakeFramed}
\def\C{\mathbb C}
\def\N{\mathbb N}
\def\Z{\mathbb Z}
\def\iff{\Leftrightarrow}
\def\R{\mathbb R}
\def\F{\mathcal F}
\def\method{\begin{leftbar}\begin{Method}}
\def\methodx{\end{Method}\end{leftbar}}
\def\thm{\begin{leftbar}\begin{Thm}}
\def\thmx{\end{Thm}\end{leftbar}}
\def\prop{\begin{Prop}}
\def\propx{\end{Prop}}
\def\defb{\begin{leftbar}\begin{Def}}
\def\defe{\end{Def}\end{leftbar}}
\def\defx{\end{Def}\end{leftbar}}
\newcommand{\supp}{\mathop{\mathrm{supp}}\nolimits}
\def\rem{\begin{Rem}}
\def\remx{\end{Rem}}
\def\prob{\begin{Prob}}
\def\probx{\end{Prob}}
\def\lem{\begin{Lemma}}
\def\lemx{\end{Lemma}}
\def\ex{\begin{Ex}}
\def\exx{\end{Ex}}
\def\cor{\begin{Cor}}
\def\corx{\end{Cor}}
\def\proof{\begin{Proof}}
\def\proofx{\end{Proof}}
\def\eq{\begin{equation}}
\def\eqx{\end{equation}}
\def\eqa{\begin{eqnarray}}
\def\eqax{\end{eqnarray}}
\def\a{\alpha}
\def\lmd{\lambda}
\def\omg{\omega}
\def\Lmd{\Lambda}
\def\Omg{\Omega}
\newcommand{\Image}{\mathop{\mathrm{Im}}\nolimits}
\newcommand{\Ker}{\mathop{\mathrm{Ker}}\nolimits}
\newcommand{\Coker}{\mathop{\mathrm{Coker}}\nolimits}
\newcommand{\vol}{\mathop{\mathrm{vol}}\nolimits}
\newcommand{\sgn}{\mathop{\mathrm{sgn}}\nolimits}
\title{平成28年度数学特別講究\\Numerical Solution of Stochastic Differential Equations}
\author{伊藤克哉}
\makeatletter
\def\thickhrulefill{\leavevmode \leaders \hrule height 1pt\hfill \kern \z@}
\renewcommand{\maketitle}{\begin{titlepage}%
    \let\footnotesize\small
    \let\footnoterule\relax
    \parindent \z@
    \reset@font
    \null
    \vskip 50\p@
    \begin{center}
      \hrule height 1pt
      \vskip 2pt 
      \hrule
      \vskip 3pt
      {\huge \bfseries \strut \@title \strut}\par
      \vskip 2pt
      \hrule
      \vskip 2pt
      \hrule height 1pt
    \end{center}
    \vskip 50\p@
    \begin{flushright}
      \Large \@author \par
    \end{flushright}
    \vfil
    \null
    \begin{flushright}
        {\small \@date}%
    \end{flushright}
  \end{titlepage}%
  \setcounter{footnote}{0}%
}
\makeatother
\usepackage{listings, jlisting}
\usepackage{color}
\definecolor{OliveGreen}{cmyk}{0.64,0,0.95,0.40}
\definecolor{colFunc}{rgb}{1,0.07,0.54}
\definecolor{CadetBlue}{cmyk}{0.62,0.57,0.23,0}
\definecolor{Brown}{cmyk}{0,0.81,1,0.60}
\definecolor{colID}{rgb}{0.63,0.44,0}
 
\lstset{
    language=Python,%プログラミング言語によって変える。
    basicstyle={\ttfamily\small},
    keywordstyle={\color{OliveGreen}},
    keywordstyle={[2]\color{colFunc}},
    keywordstyle={[3]\color{CadetBlue}},%
    commentstyle={\color{Brown}},
    %identifierstyle={\color{colID}},
    stringstyle=\color{blue},
    tabsize=2,
    frame=trBL,
    numbers=left,
    numberstyle={\ttfamily\small},
    breaklines=true,
    backgroundcolor={\color[gray]{.95}},
    captionpos=b
}
\usepackage{geometry}
\geometry{verbose,tmargin=1in,bmargin=1in,lmargin=1in,rmargin=1in}

\begin{document}
\maketitle
\tableofcontents


\chapter{Simulation and Inference for Stochastic Differential Equations}

\section{Numerical Methods for SDE}

\section{Parametric Estimation of SDE}
\[
dX_t = b_\theta (X_t) dt + \sigma_\theta(X_t) dW_t
\]
というような$\theta$によってパラメトライズされた自励系の確率微分方程式を考える.ここで,$\theta \in \Theta \subset \R^p$は$p$個のパラメータであり,$\theta_0$が推測すべき真のパラメータとする.
まずこの過程がエルゴード性を満たし,連続的である場合には容易に最尤法を適用することができる.\\
\defb
\[
dX_t = b (X_t) dt + \sigma (X_t) dW_t
\]
という自励系の確率微分方程式に対して,
\[
\tau_a = \inf \{ t\ge 0 |X_t =a \} , \tau_{ab} = \inf \{ t \ge \tau_a | X_t =b \}
\]
と定める.
過程$X_t$がrecurrentであるとは,$P(\tau_{ab} < \infty) =1$が任意の$a,b\in\R$に対して成り立つことである.\\
recurrentな過程$X_t$がpositiveであるということは,$E[\tau_{ab}] < \infty$が任意の$a,b\in\R$に対して成り立つことである.\\
\defx

\prop
\[
dX_t = b (X_t) dt + \sigma (X_t) dW_t
\]
という過程がrecurrentであることは
\[
V(x) = \int_{0}^x \exp \biggl( -2 \int_{0}^y \frac{b(u)}{\sigma(u)^2} du \biggl) dy   \to \pm\infty \ \ x\to \pm\infty
\]
ということと同値である.\\
recurrentな過程がpositiveであることは
\[
G= \int_{-\infty}^\infty \sigma(y)^{-2} \exp \biggl( 2 \int_{0}^y \frac{b(u)}{\sigma(u)^2} du \biggl) dy  < \infty
\]
であることと同値である.
\propx
Durett p.221に証明
\defb
$X_t$という過程がergodicであるとは,ある確率変数$\xi$が存在して,どのような可測関数$h(x)$に対しても
\[
\frac{1}{T} \int_0^T h(X_t) dt  \to E[h(\xi)]
\]
が成立すること.このとき,$\xi$の分布関数を$\pi$と置いて,invariant densityまたはstationary densityという.
\defx

\thm
recurrentでpositiveな過程はergodicであり,このときinvariant density\ $\pi$は
\[
\pi(x) = \frac{1}{G\sigma(x)^2} 
	\exp 
	\biggl\{ 
		2 \int_0^x \frac{b(y)}{\sigma(y)^2}dy 
	\biggl\}
\]
として表される.
\thmx
ここでergodicな過程に対する最尤法を考える.

\thm[Girsanov]
\eqa
dX_t &=& b_1(X_t) dt + \sigma(X_t) dW_t ,\ X_0 = X_0^{(1)} \\
dX_t &=& b_2(X_t) dt + \sigma(X_t) dW_t ,\ X_0 = X_0^{(2)} \\
dX_t &=& \sigma(X_t) dW_t ,\ X_0 = X_0\\
\eqax
という$t\in [0,T]$上で定義された3つのいとう過程を考える.この過程から引き起こされる確率測度を$P_1,P_2,P_3$とおく.
ここで,各$b$と$\sigma$はGlobal Lipschitzであり,Linear growthであると仮定する(ゆえに各確率微分方程式の解が存在する)\\
さらに,$f_1,f_2,f$という初期値たちの密度関数は同じサポートを持つと仮定する.\\
このとき次のようにしてこれらのRadon-Nikodym微分が存在して,次のように表される.
\[
\frac{dP_1}{dP}(X) = \frac{f_1(X_0)}{f(X_0)} 
\exp \biggl\{
	\int_0^T \frac{b_1(X_s)}{\sigma^2(X_s)} dX_s - 
	\frac{1}{2} \int_0^T \frac{ b_1^2(X_s) }{\sigma^2(X_s)} ds
	 \biggl\}
\]

\[
\frac{dP_2}{dP_1}(X) = \frac{f_2(X_0)}{f_1(X_0)} 
\exp \biggl\{
	\int_0^T \frac{b_2(X_s)-b_1(X_s)}{\sigma^2(X_s)} dX_s - 
	\frac{1}{2} \int_0^T \frac{b_2^2(X_s)- b_1^2(X_s) }{\sigma^2(X_s)} ds
	 \biggl\}
\]
	 

\thmx
任意の$\theta$に対してergodicな過程に対してはまず,quadratic variation を計算して,
\[
<X,X>_t = \lim_{n\to\infty} \sum_{k=1}^{2^n} (X_{t\wedge (k/2^n)} - X_{t\wedge ((k-1)/2^n)})^2 =
\int_0^t \sigma^2(X_s,\theta)ds 
\]
という風にして$\sigma$のパラメータを推定することができる.\\
その後,Girsanovの定理によって,次のように尤度が表される.
\[
L_T(\theta) = \frac{dP_1}{dP}(X)  
	= \exp \biggl( \int_0^T \frac{b_\theta(X_s)}{\sigma^2(X_s)} dX_s - 
	\frac{1}{2} \int_0^T \frac{ b_\theta^2(X_s) }{\sigma^2(X_s)} ds
	 \biggl)
\]
つまりこの尤度を最大にするような$\theta$を求めれば良い.\\
しかし実務上このように連続的な尤度を求めることができるのは稀であり次のような幾つかの方法で近似をしなければならない.
以下では幾つかの最尤推定法について紹介する.


\subsection{Exact likelihood inference}
前述のOrnstein-Uhlenbeck過程やGeometric Brownian motionやCIRモデルに対しては直接分布を求めて,その尤度を計算することができる.
\subsubsection{Ornstein-Uhlenbeck model}
\[
dX_t = (\theta_1 - \theta_2 X_t )  dt + \theta_3 dW_t ,\  X_0=x_0, \ \theta_1,\theta_2\in \R,\theta_3>0
\]
であるような過程に対して,条件付き密度関数は
\[
p_\theta(t,\cdot | x) \sim N(m(t,x),v(t,x))
\]
\[
m(t,x) = \frac{\theta_1}{\theta_2} + (x_0 -\frac{\theta_1}{\theta_2} ) e^{-\theta_2 t}
\]
\[
v(t,x) = \frac{\theta_3^2(1-e^{-2\theta_2 t})}{2\theta_2} 
\]
という平均と分散をもつ正規分布に従う.故にこれにより直接尤度を計算することができる.

\subsubsection{geometric Brownian motion}
次のように表されるgeometric Brownian motionを考える.
\[
dX_t = \theta_1 X_t dt + \theta_2 X_t dW_t , \ X_0 = x_0 ,\ \theta_1 \in \R,\ \theta_2 >0
\]
同様に,このモデルの条件付き密度関数は,
\[
p_\theta (t,\cdot | x)  \sim LN(m(t,x),v(t,x))
\]
\[
m(t,x)=xe^{\theta_1 t},\ v(t,x)=x^2 e^{2\theta_1 t} (e^{\theta_2^2 t} -1)
\]
というような対数正規分布によって表される.故にこれにより直接尤度を計算することができる.

\subsubsection{Cox-Ingersoll-Ross Model}
次のように表されるCox-Ingersoll-Rossモデルを考える.
\[
dX_t = (\theta_1 - \theta_2 X_t )  dt + \theta_3 \sqrt[]{X_t} dW_t ,\  X_0=x_0>0, \ \theta_1,\theta_2 \theta_3>0
\]
同様に,このモデルの条件付き密度関数は,
\[
p_\theta (t,\cdot | x) = c\exp{-u-v} \biggl(\frac{u}{v}\biggl)^{q/2} I_q(2\sqrt[]{uv}),\ x,y>0
\]
という風に表される.ただし,
\[
c = \frac{2\theta_2}{\theta_2^3(1-e^{-\theta_2 t})} , q= \frac{2\theta_1}{\theta_3^2} -1
\]
\[
u = cx\exp{-\theta_2 t} , v = cy
\]
であり,$I_q$は修正Bessel関数である.
\[
I_q(x)= \sum_{k=0}^\infty \biggl( \frac{x}{2}\biggl)^{2k+q} \frac{1}{k! \Gamma(k+q+1)}
\]
\subsection{Pseudo-likelihood methods}
SDEの離散近似スキームを使ってパスを生成し,過程の尤度を求める方法をPseudo-likelihood法という.
\subsubsection{Euler method}
\[
dX_t = b_\theta (X_t) dt + \sigma_\theta(X_t) dW_t
\]
という確率微分方程式に対してEuler-Maruyama 近似を適用すると,
\[
\Delta X = b_\theta (X_t) \Delta t + \sigma_\theta(X_t) \Delta W_t
\]
であるので,$\Delta X$はそれぞれ,平均$b_\theta(X_t)$分散$ \sigma^2_\theta(X_t) $の正規分布に従っている.
故に,Euler-Maruyama近似が強収束するとき,この対数尤度$l_n(\theta)$は
\[
l_n(\theta) = -\frac{1}{2}
	 \biggl\{ \sum_{i=1}^n \frac{X_i - X_{i-1} -b(X_{i-1},\theta)\Delta)^2}{\sigma^2 \Delta} + n\log (2\pi \sigma^2 \Delta)
	 \biggl\}
\]
という式によって表される.
またN.Yoshida(1992)によると,
\[
\hat{\sigma}^2 = \frac{1}{n\Delta} \sum_{i=1}^n (X_i - X_{i-1})^2
\]
によって$\sigma$を推定することができる.
ここで,最初に述べたExact likelihoodとEuler approximationの性質を見てみる.
\[
dX_t = (\theta_1 - \theta_2 X_t )  dt + \theta_3 dW_t ,\  X_0=x_0, \ \theta_1,\theta_2\in \R,\theta_3>0
\]
というOrnstein-Uhlenbeck過程に対して2つの方法を適用しすると,
\[
m(\Delta,x) = x\exp(-\theta_2\Delta) + \frac{\theta_1}{\theta_2} (1- \exp(-\theta_2 \Delta)) , 
\ v(\Delta,x) = \frac{\theta_3^2(1-\exp^{-2\theta_2 \Delta})}{2\theta_2} 
\]
という平均と分散が得られ,またEuler approximationを行うと,
\[
m(\Delta,x) = x(1-\theta_2 \Delta) + \theta_1 \Delta , \ v(\Delta,x) = \theta_3^2 \Delta
\]
という平均と分散が得られる.これらは,$\Delta \to 0$では一致する.
一方で,
\[
dX_t = \theta_1 X_t dt + \theta_2 X_t dW_t , \ X_0 = x_0 ,\ \theta_1 \in \R,\ \theta_2 >0
\]
というgeometric Brownian motionを考える.この時,上の式出てきた最尤法を当てはめると,
\[
\hat{\theta}_{1,n} = \frac{1}{n\Delta} \sum_{i=1}^n \biggl( \frac{X_i}{X_{i-1}} -1 \biggl),
\hat{\theta}_{2,n} = \frac{1}{n\Delta} \sum_{i=1}^n \biggl( \frac{X_i}{X_{i-1}} -1  - \hat{\theta}_{1,n}^2 \Delta \biggl),
\]
という最尤推定値を得る.ここで,$\Delta$が無視できないほど大きい場合を考えると,
\[
\hat{\theta}_{1,n} \to \frac{1}{\Delta} (e^{\theta_1 \Delta} -1 ) \neq \theta_1,
\hat{\theta}_{2,n} \to \frac{1}{\Delta}e^{2 \theta_1 \Delta}  (e^{\theta_2^2 \Delta} -1 ) \neq \theta_2
\]
という確率収束が得られるので,$\Delta$の影響は大きい.

\subsubsection{Elerian method}
Milstein近似により,
\[
\Delta X_t = b(t,X_t)\Delta t + \sigma(t,X_t) \Delta W + \frac{1}{2} \sigma(t,X_t)\sigma_x(t,X_t)((\Delta W)^2 - \Delta t)
\]
という近似式が得られている.このとき次の定理が成り立つ
\thm
Milstein近似がオーダー$1$で強収束しているとき
\[
p(t,y|x) = \frac{z^{-1/2} \cosh (\sqrt[]{Cz}) }{|A|\sqrt[]{2\pi}} \exp(-\frac{C+z}{2})
\]
として得られる.ただし,
\[
A =\frac{\sigma(x)\sigma_x(x) t}{2} , B = -\frac{\sigma(x)}{2\sigma_x(x)}+x+b(x)t-A,
\]
\[
z = \frac{y-B}{A} , C = \frac{1}{\sigma^2_x(x) t}
\]
\thmx
\subsubsection{Shoji-Ozaki method}
Shoji-Ozaki法は,
\[
dX_t = b (t,X_t) dt + \sigma dW_t
\]
という確率微分方程式の解に対して,これを$[s,s+\Delta s)$上で局所的に近似することによって,
\[
dX_t = (L_s X_t + tM_s + N_s )dt+ \sigma  dW_t
\]
ただし,
\[
L_s = b_x(s,X_s) , \ M_s = \frac{\sigma^2}{2} b_{xx} (s,X_s) + b_t(s,X_s)
\]
\[
N_s = b(s,X_s) - X_s b_x(s,X_s) - sM_s
\]
に対して,$Y_t = e^{-L_s t } X_t $変換することによって,
\[
Y_t = Y_s + \int_s^t (M_s u + N_s ) e^{-L_s u} du + \sigma \int_{s}^t e^{-L_s u} dW_u
\]
という確率微分方程式が得られるが,これを部分的に解いて,$X_t$に戻す事によって,
\[
X_{s+\Delta s} = A(X_s)X_s + B(X_s) Z
\]

ただし,
\eqa
A(X_s) &=& 1 + \frac{b(s,X_s)}{X_s L_s} (e^{L_s \Delta s} -1 ) +
\frac{M_s}{X_s L_s^2} (e^{L_s \Delta s} -1  - L_s \Delta s) \\
B(X_s) &=& \sigma\ \sqrt[]{ \frac{\exp{2L_s \Delta s} -1 }{2L_s}},\ Z\sim N(0,1)
\eqax

である.これを利用することによって,尤度関数を計算することができる.

\subsection{Approximated likelihood methods}
次に,パスを近似するのではなく直接尤度を近似する方法について考える.
\subsubsection{Kessler method}
ここで,
\[
dX_t = b_\theta (X_t) dt + \sigma_\theta(X_t) dW_t
\]
というような自励系のergodicな確率過程を考える.\\
ここで
\[
t^n_i = i \Delta_n
\]
とおいて,この幅で離散化することを考える.\\
ここで,Ito-Taylor展開を考え,
\[
r_l(\Delta_n,x,\alpha) =\sum_{i=0}^l
	\frac{\Delta_n^i}{i!}L^i_\alpha f(x)
\]
という第$l$項まで展開したものを考える.このとき,$n\Delta_n^p \to 0$であるとして,
$k_0 = [p/2]$と定めると次のようにして,尤度を定めることができる.
\[
l_{p,n}(\alpha) =
\sum_{i=1}^n
	\frac{(X_{t^n_i} - r_{k_0}(\Delta_n,X_{t^n_{i-1}},\alpha))^2}{\Delta_n \sigma^2_{i-1}(\theta)}
	\biggl\{
		1+\sum_{j=1}^{k_0} \Delta_n^j d_j(X_{t^n_{i-1}},\alpha)
	\biggl\}	
+ \sum_{i=1}^n
	\biggl\{
		\log c_{i-1}(\theta) + 
		\sum_{j=1}^{k_0} + \Delta_n^j e_j(X_{t^n_{i-1}},\alpha)
	\biggl\}
\]

ただし,$d_j,e_j$は特別な関数のTaylor展開をしたときの係数である.

\subsubsection{Simulated likelihood method}
$p_\theta(\Delta,y|x)$を求めることを考える.$\delta \ll \Delta$とする.このとき,
\[
p_\theta(\Delta,y|x) = 
\int p_\theta(\delta,y|z)p_\theta(\Delta - \delta,z|x)dz 
= E_z[p_\theta(\delta,y|z)|\Delta-\delta]
\]
であることがわかる.ここで,右の期待値をMonte Carlo法によって求めると,
$p_\theta(\Delta - \delta,z|x)$は正規分布であるので,
\[
\hat{p}_\theta^{(N,M)}(\Delta,X_{t+\Delta}|x)
= \frac{1}{M}\sum_{i=1}^M \phi_\theta(\delta,X_{t+\Delta}|z_i)
\]
として,確率密度関数を求めることができる.

\subsubsection{Hermite polynomials expansion of the likelihood}

\[
H_j(z) = e^{\frac{z^2}{2}} \frac{d^j}{dz^j} e^{-\frac{z^2}{2}} 
\]
という修正Hermite多項式を考える.
ここで,今までと同様に$X_t$という伊藤過程に対して,
\[
F(\gamma) = \int_0^\gamma \frac{du}{\sigma_\theta(u)}
\]
というLamperti変換を加えることによって,まず
\[
dY_t = \mu_Y(Y_t,\theta)dt + dW_t
\]
という風に変換する.
その次に,
\[
Z = \Delta^{-1/2} (Y-y_0)
\]
という変換を施して,
\[
p_Z(\Delta,z|y_0,\theta) = \Delta^{1/2} p_Y(\Delta,\Delta^{1}{2}z+y_0|y_0,\theta)
\]
という確率密度関数を求めることを考える.ここで,$H_j(z)/\sqrt[]{j!}$は$L^2$空間の直交基底であることを使うと,
$p_Z$を分解することができる.そして,その展開を途中で打ち切ることによって,尤度の近似を行う.

\end{document}
